\documentclass[10pt]{article}

\usepackage{fullpage}
\usepackage{times}
\usepackage{listings}
\lstdefinelanguage{chapel}
  {
    morekeywords={
      align, atomic,
      begin, bool, break, by,
      class, cobegin, coforall, complex, config, const, continue,
      delete, dmapped, do, domain,
      else, enum, extern,
      false, for, forall,
      if, imag, in, index, inline, int, inout, iter,
      label, let, local, locale,
      module,
      new, nil,
      on, opaque, otherwise, out,
      param, proc,
      range, real, record, reduce, ref, return,
      scan, select, serial, single, sparse, string, subdomain, sync,
      then, true, tuple, type,
      uint, union, use,
      var,
      when, where, while,
      yield,
      zip
    },
    sensitive=false,
    mathescape=true,
    morecomment=[l]{//},
    morecomment=[s]{/*}{*/},
    morestring=[b]",
}

\lstset{
    basicstyle=\footnotesize\ttfamily,
    keywordstyle=\bfseries,
    commentstyle=\em,
    showstringspaces=false,
    flexiblecolumns=false,
    numbers=left,
    numbersep=5pt,
    numberstyle=\tiny,
    numberblanklines=false,
    stepnumber=0,
    escapeinside={(*}{*)},
    language=chapel,
  }

%\newcommand{\chpl}[1]{\lstinline[language=chapel,basicstyle=\ttfamily,keywordstyle=\bfseries]!#1!}
\newcommand{\chpl}[1]{\lstinline[language=chapel,basicstyle=\small\ttfamily,keywordstyle=]!#1!}
\newcommand{\varname}[1]{\emph{#1}}
\newcommand{\typename}[1]{\emph{#1}}
\newcommand{\fnname}[1]{\chpl{#1}}

\lstnewenvironment{chapel}{\lstset{language=chapel,xleftmargin=2pc,stepnumber=0}}{}
\lstnewenvironment{invisible}{\lstset{language=chapel,xleftmargin=2pc,stepnumber=0,keywordstyle=\bfseries\color{white},basicstyle=\small\ttfamily\color{white}}}{}
\lstnewenvironment{chapel0}{\lstset{language=chapel,stepnumber=0}}{}

\lstnewenvironment{numberedchapel}{\lstset{language=chapel,xleftmargin=15pt,stepnumber=1}}{}

\lstnewenvironment{chapelcode}{\lstset{language=chapel,stepnumber=1}}{}

\lstnewenvironment{commandline}{\lstset{keywordstyle=,xleftmargin=2pc}}{}

\lstnewenvironment{protohead}{\lstset{language=chapel,xleftmargin=0pc,belowskip=-10pt,stepnumber=0}}{}

\newenvironment{protobody}{\begin{description}\item[\quad\quad] }{\end{description}}


\pagestyle{empty}

\begin{document}
\lstset{language=chapel}

\section*{Histogram Exercise}

\section{Serial Histogram}
Using histogram.chpl as a starting point, fill in the function
\begin{chapel}
computeHistogram(X: [] real, Y: [] int)
\end{chapel}
This function takes two arguments.  \chpl{X} is an array of random
numbers defined over \chpl{1..numNumbers}.  \chpl{Y} is an empty array
that should contain the histogram data.  It is defined over
\chpl{1..numBuckets}.

The output should look something like:
\begin{quote}
\begin{footnotesize}
\begin{verbatim}
> a.out
Running Histogram Example
 Number of Random Numbers =        8
 Number of Buckets        =       10

Random Numbers

0.585924 0.59617 0.491852 0.788843 0.354106 0.031778 0.469706 0.4702

Histogram computed in 1e-06 seconds

     X     
     XX    
 X  XXX X  
+----------+
 Raw Data
  0.00 - 0.10: 1
  0.10 - 0.20: 0
  0.20 - 0.30: 0
  0.30 - 0.40: 1
  0.40 - 0.50: 3
  0.50 - 0.60: 2
  0.60 - 0.70: 0
  0.70 - 0.80: 1
  0.80 - 0.90: 0
  0.90 - 1.00: 0
\end{verbatim}
\end{footnotesize}
\end{quote}

To change the problem size, set the configuration variable on the
command line as follows:
\begin{quote}
\begin{footnotesize}
\begin{verbatim}
> a.out --numNumbers=1000
\end{verbatim}
\end{footnotesize}
\end{quote}
You can also change the number of buckets via \chpl{numBuckets} and
you can avoid printing the random numbers by setting
\chpl{printRandomNumbers} to \chpl{false}.  Try the following run:
\begin{quote}
\begin{footnotesize}
\begin{verbatim}
> a.out --printRandomNumbers=false --numNumbers=1000000
\end{verbatim}
\end{footnotesize}
\end{quote}

\section{Parallel Histogram}
Now add another configuration constant with the line
\begin{chapel}
config const numThreads: int = 4;
\end{chapel}
Then use \chpl{numThreads} in a function that computes the histogram
with that many threads.  Hint: use a \chpl{coforall} statement to
create the tasks and a sync variable to ensure that the code is
race-free.

Does your parallel code run faster than your serial code?  What about
at larger problem sizes?

\section{Normal Histogram}
Now add another array of random numbers and compute the histogram of
the element-wise averages.  You can store the averages in the first
array for simplicity.

\end{document}
