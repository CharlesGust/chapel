\sekshun{Classes}
\label{Classes}
\index{class@\chpl{class}}
\index{classes}

Classes are data structures with associated state and functions. Storage for
\index{classes!instances}
\index{objects}
a class instance, or object, is allocated independently of the scope of
the variable that refers to it.
An object is created by calling a class constructor
(\rsec{Class_Constructors}), which allocates storage, initializes it,
and returns a reference to the newly-created object.
Storage can be reclaimed by deleting the object (\rsec{Class_Delete}).

A class declaration (\rsec{Class_Declarations}) generates a class
type (\rsec{Class_Types}).  A variable of a class type can refer to an
instance of that class or any of its derived classes.

A class is generic if it has generic fields. Generic classes and fields
are discussed in~\rsec{Generic_Types}.

\section{Class Declarations}
\label{Class_Declarations}
\index{classes!declarations}
\index{class@\chpl{class}}

A class is defined with the following syntax:
\begin{syntax}
class-declaration-statement:
  external-class-declaration-statement
  simple-class-declaration-statement

simple-class-declaration-statement:
  `class' identifier class-inherit-list[OPT] { class-statement-list[OPT] }

class-inherit-list:
  : class-type-list

class-type-list:
  class-type
  class-type , class-type-list

class-statement-list:
  class-statement
  class-statement class-statement-list

class-statement:
  variable-declaration-statement
  method-declaration-statement
  type-declaration-statement
  empty-statement
\end{syntax}

A \sntx{class-declaration-statement} defines a new type symbol
specified by the identifier.  Classes inherit data and functionality
from other classes %and/or records
if the \sntx{inherit-type-list} is specified.
Inheritance is described in~\rsec{Inheritance}.

\begin{future}
Record inheritance is not currently implemented.  When enabled, it will have the
same effect as creating a field of that record type within the class, except
that all of the record's field and method names will appear within the scope of
the inheriting class.
\end{future}

The body of a class declaration consists of a sequence of statements
where each of the statements either defines a variable (called a
field), a procedure or iterator (called a method), or a type alias.  In addition, empty
statements are allowed in class declarations, and they have no effect.

If a class declaration contains a type alias or a parameter field, or it contains a variable or
constant without a specified type and without an initialization
expression, then it declares a generic class type.  Generic classes are described
in~\rsec{Generic_Types}.

If the \chpl{extern} keyword appears before the \chpl{class} keyword, then an
external class type is declared.  An external class type declaration must not
contain a \sntx{class-inherit-list}.  An external class is used within Chapel
for type and field resolution, but no corresponding backend definition is
generated.  It is presumed that the definition of an external class is supplied
by a library or the execution environment.  See the chapter on interoperability
(\rsec{Interoperability}) for more information on external classes.
% External class inheritance is not currently supported.

\subsection{Class Types}
\label{Class_Types}
\index{classes!class type}
\index{class type}

A class type is given simply by the class name for non-generic classes.
Generic classes must be instantiated to serve as a fully-specified
type, for example to declare a variable.  This is done with
type constructors, which are defined in Section~\ref{Type_Constructors}.

\begin{syntax}
class-type:
  identifier
  identifier ( named-expression-list )
\end{syntax}

A class type may appear in inheritance lists of
other class declarations.

\subsection{Class Values}
\label{Class_Values}
\index{classes!class value}
\index{class value}

A class value is either a reference to an instance of a class
or \chpl{nil} (\rsec{Class_nil_value}). Class instances can be created
using the \chpl{new} operator (\rsec{Class_New}) and deleted using
the \chpl{delete} operator (\rsec{Class_Delete}).

For a given class type, a legal value of that type is a reference to
an instance of either that class or a class inheriting, directly or
indirectly, from that class.
\chpl{nil} is a legal value of any class type.

The default value of a class type is \chpl{nil}.

\begin{chapelexample}{declaration.chpl}
\begin{chapel}
class C { }
var c : C;	// c has the class type C, initialized with the value nil.
c = new C();	// Now c refers to an object of type C.
var c2 = new C();	// The type of c2 is also C.
class D : C {}	// Class D is derived from C.
c = new D();	// Now c refers to an object of type D.
\end{chapel}
\begin{chapeloutput}
\end{chapeloutput}
When the variable \chpl{c} is declared, it initially has the value
of \chpl{nil}.  The next statement assigned to it an instance of the
class \chpl{C}.  The declaration of variable \chpl{c2} shows that these steps can
be combined.  The type of \chpl{c2} is also \chpl{C}, determined implicitly from
the the initialization expression.  Finally, an object of type \chpl{D} is created and
assigned to \chpl{c}.  The object previously referenced by \chpl{c} is no longer
referenced anywhere. It could be reclaimed by the garbage collector.
\end{chapelexample}

\subsection{Class Fields}
\label{Class_Fields}
\index{classes!fields}
\index{fields}

Variable declarations within a class declaration define
fields within that class.  Parameter fields make a class generic.
Variable fields define the storage associated with a class instance.

\begin{chapelexample}{defineActor.chpl}
The code
\begin{chapel}
class Actor {
  var name: string;
  var age: uint;
}
\end{chapel}
\begin{chapeloutput}
\end{chapeloutput}
defines a new class type called \chpl{Actor} that has two fields: the
string field \chpl{name} and the unsigned integer field \chpl{age}.
\end{chapelexample}

Field access is described in \rsec{Class_Field_Accesses}.

\subsection{Class Methods}
\label{Class_Methods}
\index{classes!methods}

A class method is a procedure or iterator that is bound to a class.

Methods are declared with the following syntax:
\begin{syntax}
method-declaration-statement:
  linkage-specifier[OPT] proc-or-iter param-clause[OPT] type-binding function-name argument-list[OPT] 
    return-intent[OPT] return-type[OPT] where-clause[OPT] function-body

proc-or-iter:
  `proc'
  `iter'

param-clause:
  `param'

type-binding:
  identifier .
\end{syntax}
If a method is declared within the lexical scope of a class, record,
or union, the type binding can be omitted and is taken to be the
innermost class, record, or union that the method is defined in.  If a
method declaration contains the optional \sntx{param-clause}, it
implies that it can only be applied to param objects of the given type
binding.

Method calls are described in \rsec{Class_Method_Calls}.

\subsection{Nested Classes}
\label{Nested_Classes}
\index{classes!nested classes}
\index{nested classes}

A class or record defined within another class is a nested class (or record).

Nested classes or records can refer to fields and methods in the outer class (or
record) implicitly, or explicitly by means of an \chpl{outer} reference.

A nested class (or record) can be referenced only within
its immediately enclosing class (or record).

\section{Inheritance}
\label{Inheritance}
\index{inheritance}
\index{classes!inheritance}
\index{derived class}
\index{classes!derived class}

A \emph{derived} class can inherit from one or more other classes by
listing those classes in the derived class declaration.
%REVIEW: vass: do we want to allow multiple base classes in the spec?
%REVIEW: vass: need to define "base" class and add it to index, if useful
When inheriting from multiple base classes, only one of the base classes
may contain fields.  The other classes can only define methods.  Note
that a class can still be derived from a class that contains fields
which is itself derived from a class that contains fields.  

%REVIEW: vass: the below ("tree") does not match the above ("multiple base classes")
These restrictions on inheritance induce a class hierarchy which has the form of
a tree.  A variable referring to an instance of class \chpl{C} can be
cast to any type that is an ancestor of \chpl{C}.  Note that casts to more- and
less-derived classes are both permitted.

\begin{future}
A derived class may also incorporate any number of records by listing them
in the derived class declaration.
As with record inheritance, this has the effect of injecting the record's fields and
methods into the new class type.  Record inheritance does not induce a
well-defined class hierarchy.  See~\rsec{Record_Inheritance} for details.
\end{future}

\subsection{The object Class}
\label{The_object_Class}

All classes are derived from the \chpl{object} class, either directly or
indirectly.  If no class name appears in the inheritance list, the class derives
implicitly from \chpl{object}.  Otherwise, a class derives from \chpl{object}
indirectly through the class or classes it inherits.  A variable of type \chpl{object}
can hold a reference to an object of any class type. 

\subsection{Accessing Base Class Fields}
\label{Accessing_Base_Class_Fields}

A derived class contains data associated with the fields in its base
classes.  The fields can be accessed in the same way that they are
accessed in their base class unless the getter or setter method is
overridden in the derived class, as discussed
in~\rsec{Overriding_Base_Class_Methods}.

\subsection{Derived Class Constructors}
\label{Derived_Class_Constructors}
\index{constructors!for derived classes}

The default initializer of a derived class automatically calls the default
initializer of each of its base classes %and records.  
The same is not true for constructors:
To initialize inherited fields to anything other than its default-initialized
value, a constructor defined in a derived class must either call base class
constructors or manipulate those base-class fields directly.

\begin{openissue}
The syntax for calling a base-class constructor from a derived-class constructor
has not yet been defined.

There is an expectation that a more standard way
of chaining constructor calls will be supported.
\end{openissue}

\subsection{Shadowing Base Class Fields}
\label{Shadowing_Base_Class_Fields}

A field in the derived class can be declared with the same name as a
field in the base class.  Such a field shadows the field in the base
class in that it is always referenced when it is accessed in the
context of the derived class.  

\begin{openissue}
There is an expectation that there will
be a way to reference the field in the base class but this is not
defined at this time.
\end{openissue}

\subsection{Overriding Base Class Methods}
\label{Overriding_Base_Class_Methods}
\index{dynamic dispatch}

If a method in a derived class is declared with a
signature identical to that of a method in a base class, then it is said to override the
base class's method.  Such a method is a candidate for dynamic
dispatch in the event that a variable that has the base class type
references an object that has the derived class type.

The identical signature requires that the names, types, and order of
the formal arguments be identical. The return type of the overriding
method must be the same as the return type of the base class's method,
or must be a subclass of the base class method's return type.

Methods without parentheses are not candidates for dynamic dispatch.
\begin{rationale}
Methods without parentheses are primarily used for field accessors.  
A default is created if none is specified.  The field accessor
should not dispatch dynamically since that would make it
impossible to access a base field within a base method should that
field be shadowed by a subclass.
\end{rationale}

\subsection{Inheriting from Multiple Classes}
\label{Inheriting_from_Multiple_Classes}
\index{multiple inheritance}
\index{classes!inheritance!multiple}

%REVIEW: vass: need to decide as a group whether we want this in the spec
A class can be derived from multiple base classes provided that only
one of the base classes contains fields either directly or from base
classes that it is derived from.  The methods defined by the other
base classes can be overridden.  This provides functionality similar to the C\#
concept of interfaces.

\begin{openissue}
It is an open question whether the language will support \chpl{interface}
declarations and multiple inheritance. This is currently under study
at the University of Colorado (Boulder).
\end{openissue}

\subsection{The {\em nil} Value}
\label{Class_nil_value}
\index{classes!nil}
\index{nil@\chpl{nil}}

Chapel provides \chpl{nil} to indicate the absence of a reference to
any object.  \chpl{nil} can be assigned to a variable of any class
type.  Invoking a class method or accessing a field of the \chpl{nil}
value results in a run-time error.

\subsection{Default Initialization}
\label{Default_Initialization}
\index{classes!initialization}
\index{initialization!classes}
\index{classes!initialization!default}
\index{classes!default initialization}
\index{default initialization!classes}
\index{initialization!classes!default}

%REVIEW: vass:this does not correspond to the current implementation:
% * when invoking the compiler-generated constructor, default initialization
%   does not happen;
% * in many cases (?), fields' initialization expressions are not invoked.
When an instance of a class (an object) is created, it is brought to a
known and legal state first, before it can be accessed or operated upon.
This is done through default initialization.

An object is default-initialized by initializing all of its fields in
the order of the field declarations within the class. Fields inherited
from a superclass are initialized before fields declared in current class.

If a field in the class is declared with an initialization expression, that
expression is used to initialize the field.  Otherwise, the field is
initialized to the default value of its type
(\rsec{Default_Values_For_Types}).

\section{Class Constructors}
\label{Class_Constructors}
\label{Class_New}
\index{classes!constructors}
\index{classes!new}
\index{new!classes}
\index{new@\chpl{new}}
\index{constructors}

Class instances are created by invoking class constructors.
A class constructor is a method with the same name as the class.
It is invoked by the \chpl{new} operator, where the
class name and constructor arguments are preceded with the
\chpl{new} keyword.

When the constructor is called, memory is allocated to store
a class instance, the instance undergoes default initialization, and
then the constructor method is invoked on this newly-created
instance.

If the program declares a class constructor method,
it is a user-defined constructor.  
If the program declares no constructors for a class,
a compiler-generated constructor for that class is created automatically.

\subsection{User-Defined Constructors}
\label{User_Defined_Constructors}
\index{classes!constructors!user-defined}
\index{user-defined constructors}
\index{constructors!user-defined}

A user-defined constructor is a constructor method explicitly declared
in the program.  A constructor declaration has the same
syntax as a method declaration, except that the name of the function matches
the name of the class, and there is no return type specifier.

A constructor for a given class is called by placing the \chpl{new} operator
in front of the class name.  Any constructor arguments follow the class name in a
parenthesized list.

\begin{syntax}
constructor-call-expression:
  `new' class-name ( argument-list )

class-name:
  identifier
\end{syntax}

When a constructor is called, the usual function resolution mechanism
(\rsec{Function_Resolution}) is applied to determine which
user-defined constructor to invoke.

\begin{chapelexample}{simpleConstructors.chpl}
The following example shows a class with two constructors:
\begin{chapel}
class MessagePoint {
  var x, y: real;
  var message: string;

  proc MessagePoint(x: real, y: real) {
    this.x = x;
    this.y = y;
    this.message = "a point";
  }

  proc MessagePoint(message: string) {
    this.x = 0;
    this.y = 0;
    this.message = message;
  }
}  // class MessagePoint

// create two objects
var mp1 = new MessagePoint(1.0,2.0);
var mp2 = new MessagePoint("point mp2");
\end{chapel}
\begin{chapelpost}
writeln(mp1);
writeln(mp2);
\end{chapelpost}
\begin{chapeloutput}
{x = 1.0, y = 2.0, message = a point}
{x = 0.0, y = 0.0, message = point mp2}
\end{chapeloutput}
The first constructor lets the user specify the initial coordinates
and the second constructor lets the user specify the initial message
when creating a MessagePoint.
\end{chapelexample}

Constructors for generic classes (\rsec{Generic_Types}) handle certain
arguments differently and may need to satisfy additional
requirements. See Section~\ref{Generic_User_Constructors} for details.

\subsection{The Compiler-Generated Constructor}
\label{The_Compiler_Generated_Constructor}
\index{classes!constructors!compiler-generated}
\index{constructors!compiler-generated}
\index{compiler-generated constructors}

A compiler-generated constructor for a class is created automatically
if there are no constructors for that class in the program.
The compiler-generated constructor has one argument for every field in the class,
each of which has a default value equal to the field's initializer (if present) or default value of the field's type (if not).
The list of fields (and hence arguments) includes fields inherited from superclasses, type aliases
and parameter fields, if any.
The order of the arguments in the argument list matches the order of the field declarations
within the class, with the arguments for a superclass's fields occurring
before the arguments for the fields declared in current class.

Generic fields are treated separately
which are discussed in Section~\rsec{Generic_Compiler_Generated_Constructors}.

When invoked, the compiler-generated constructor initializes each field in the class to the
value of the corresponding actual argument.  

%
% BLC: I'm not comfortable enough with this to put it into the release;
% I've made minor mods to the above to present what I hope is a sufficiently
% consistent story.  We're going to need to revisit this whole section upon
% finishing constructors anyway
%
%% In contrast to an actual function
%% call (including calls to user-defined constructors), the compiler-generated
%% constructor only initializes fields for which arguments are supplied.  All other
%% fields retain their default-initialized values.  In this sense the
%% compiler-generated constructor is actually a family of constructors, one for
%% each permissible combination of named and unnamed arguments.\footnote{This makes
%% it clear why the compiler-generated constructor is visible only if no
%% user-defined constructors are present.}

%REVIEW: vass
% I am unhappy with having to present this as "not an actual function call".
% Background: two ways for presenting constructors have been used.
% Previously:
% - compiler-generated constructor is always there
% - it is implicitly invoked at the beginning of a user-defined constructor
% - there is no default initialization
% Currently
% - compiler-generated constructor exists only if no user-defined constructors
% - default initialization always happens before a constructor is invoked
% - compiler-generated constructor initializes only the fields for which
%   its invocation provides actual arguments
%   (hence it is magic, unlike a function call)
% 
% The motivation for this refactoring is:
% - give the reader a firm understanding that an object is zeroed-out
%   before they can do anything with it, and
% - eliminate compiler-generated constructor when there is a user-defined one
% This leads us to the "unlike a function call" semantics here -
% to avoid duplicate invocation of fields' default initializers
% (which otherwise would happen once during the object's default initialization
% and the second time upon invoking the compiler-generated constructor).
% But overall I do not feel that this refactoring is a win, especially
% because we now need to define the magic (and I am not sure presently
% it is defined formally enough).

\begin{chapelexample}{defaultConstructor.chpl}
Given the class
\begin{chapel}
class C {
  var x: int;
  var y: real = 3.14;
  var z: string = "Hello, World!";
}
\end{chapel}
\begin{chapelpost}
var c1 = new C();
var c2 = new C(2);
var c3 = new C(z="");
var c4 = new C(2, z="");
var c5 = new C(0, 0.0, "");
writeln((c1, c2, c3, c4, c5));
\end{chapelpost}
\begin{chapeloutput}
({x = 0, y = 3.14, z = Hello, World!}, {x = 2, y = 3.14, z = Hello, World!}, {x = 0, y = 3.14, z = }, {x = 2, y = 3.14, z = }, {x = 0, y = 0.0, z = })
\end{chapeloutput}
there are no user-defined constructors for \chpl{C}, so \chpl{new} operators
will invoke \chpl{C}'s compiler-generated constructor. The \chpl{x} argument
of the compiler-generated constructor has the default value \chpl{0}.
The \chpl{y} and \chpl{z} arguments have the default values \chpl{3.14} and
\chpl{"Hello, World\!"}, respectively.

\chpl{C} instances can be created by calling the compiler-generated constructor as follows:
\begin{itemize}
\item The call \chpl{new C()} is equivalent to \chpl{C(0,3.14,"Hello, World\!")}.
\item The call \chpl{new C(2)} is equivalent to \chpl{C(2,3.14,"Hello, World\!")}.
\item The call \chpl{new C(z="")} is equivalent to \chpl{C(0,3.14,"")}.
\item The call \chpl{new C(2, z="")} is equivalent to \chpl{C(2,3.14,"")}.
\item The call \chpl{new C(0,0.0,"")} specifies the initial values for all fields explicitly.
\end{itemize}
\end{chapelexample}

\section{Field Accesses}
\label{Class_Field_Accesses}
\index{classes!field access}
\index{field access}

The field in a class is accessed via a field access expression.

\begin{syntax}
field-access-expression:
  receiver-clause[OPT] identifier

receiver-clause:
  expression .
\end{syntax}

\index{classes!receiver}
\index{receiver}
% TODO: these are the rules to determine the receiver - move them
% to a separate section so they can be uniformly referenced from everywhere.
The receiver-clause specifies the \emph{receiver}, which is the class
instance whose field is being accessed.
The receiver clause can be omitted when the field access is within a method.
In this case the receiver is the method's receiver \rsec{The_em_this_Reference}.
The receiver clause can also be omitted when the field access is within
a class declaration. In this case the receiver is the instance
being implicitly defined or referenced.

The identifier in the field access expression indicates which field is accessed.

% TODO: rephrase all this in terms of the getter methods

A field can
be modified via an assignment statement where the left-hand side of
the assignment is a field access expression.
Accessing a parameter field returns a parameter.

\begin{chapelexample}{useActor1.chpl}
Given a variable \chpl{anActor} of type \chpl{Actor} as defined above,
the code
\begin{chapelpre}
use defineActor;
var anActor = new Actor(name="Tommy", age=26);
\end{chapelpre}
\begin{chapel}
var s: string = anActor.name;
anActor.age = 27;
\end{chapel}
\begin{chapelpost}
writeln((s, anActor));
\end{chapelpost}
\begin{chapeloutput}
(Tommy, {name = Tommy, age = 27})
\end{chapeloutput}
reads the field \chpl{name} and assigns the value to the variable
\chpl{s}, and assigns the field \chpl{age} in the object
\chpl{anActor} the value \chpl{27}.
\end{chapelexample}

\subsection{Variable Getter Methods}
\label{Getter_Methods}
\index{classes!getter method}
\index{getter method}

All field accesses are resolved via getters that are variable
methods~(\rsec{Variable_Functions}) defined in the class with the same
name as the field.  The default getter is defined to simply return the
field if the user does not redefine it.

\begin{chapelexample}{getterSetter.chpl}
In the code
\begin{chapel}
class C {
  var setCount: int;
  var x: int;
  proc x var {
    if setter then
      setCount += 1;
    return x;
  }
}
\end{chapel}
\begin{chapelpost}
var c = new C();
c.x = 1;
writeln(c.x);
c.x = 2;
writeln(c.x);
c.x = 3;
writeln(c.x);
writeln(c.setCount);
\end{chapelpost}
\begin{chapeloutput}
1
2
3
3
\end{chapeloutput}
an explicit variable getter method is defined for field \chpl{x}.  It
returns the field \chpl{x} and increments another field that records
the number of times x was assigned a value.
\end{chapelexample}

\section{Class Method Calls}
\label{Class_Method_Calls}
\index{classes!method calls}
\index{method calls}

A method is invoked with a method call, which is similar to a non-method
call expression.

\begin{syntax}
method-call-expression:
  receiver-clause[OPT] expression ( named-expression-list )
  receiver-clause[OPT] expression [ named-expression-list ]
  receiver-clause[OPT] parenthesesless-function-identifier
\end{syntax}

The receiver-clause (or its absence) specifies the method's receiver
\rsec{The_em_this_Reference} in the same way it does for field accesses
\rsec{Class_Field_Accesses}.

\begin{chapelexample}{defineMethod.chpl}
A method to output information about an instance of the \chpl{Actor}
class can be defined as follows:
\begin{chapelpre}
use useActor1;
\end{chapelpre}
\begin{chapel}
proc Actor.print() {
  writeln("Actor ", name, " is ", age, " years old");
}
\end{chapel}
\begin{chapelpost}
anActor.print();
\end{chapelpost}
\begin{chapeloutput}
(Tommy, {name = Tommy, age = 27})
Actor Tommy is 27 years old
\end{chapeloutput}
This method can be called on an instance of the \chpl{Actor}
class, \chpl{anActor}, with the call expression \chpl{anActor.print()}.
\end{chapelexample}

The actual arguments supplied in the method call are bound to the
formal arguments in the method declaration following the rules specified for
procedures (\rsec{Functions}). The exception is the receiver
\rsec{The_em_this_Reference}.

\subsection{The Method Receiver and the {\em this} Argument}
\label{The_em_this_Reference}
\index{this@\chpl{this}}
\index{classes!receiver}
\index{receiver}

A method's \emph{receiver} is its implicit argument, 
the class instance on which the method is invoked.
Its type is the class in which the method is defined.
The receiver actual argument is specified
by the receiver-clause of a method-call-expression
as specified in \rsec{Class_Field_Accesses}.
% TODO: specify how the receiver affects the choice of the method.
The receiver formal argument can be referred to within the method
using the identifier \chpl{this}.

\begin{chapelexample}{implicitThis.chpl}
Let class \chpl{C}, method \chpl{foo}, and function \chpl{bar} be
defined as
\begin{chapel}
class C {
  proc foo() {
    bar(this);
  }
}
proc bar(c: C) { writeln(c); }
\end{chapel}
\begin{chapelpost}
var c1: C = new C();
c1.foo();
\end{chapelpost}
\begin{chapeloutput}
{}
\end{chapeloutput}
Then given an instance of \chpl{C} called \chpl{c1}, the method
call \chpl{c1.foo()} results in a call to \chpl{bar} where the argument
is \chpl{c1}.
\end{chapelexample}

\section{The {\em this} Method}
\label{The_em_this_Method}
\index{this@\chpl{this}}
\index{classes!indexing}

A procedure method declared with the name \chpl{this} allows a class to be
``indexed'' similarly to how an array is indexed.  Indexing into a
class instance has the semantics of calling a method
named \chpl{this}.  There is no other way to call a method
called \chpl{this}.  The \chpl{this} method must be declared with
parentheses even if the argument list is empty.

\begin{chapelexample}{thisMethod.chpl}
In the following code, the \chpl{this} method is used to create a
class that acts like a simple array that contains three integers
indexed by 1, 2, and 3.
\begin{chapel}
class ThreeArray {
  var x1, x2, x3: int;
  proc this(i: int) var {
    select i {
      when 1 do return x1;
      when 2 do return x2;
      when 3 do return x3;
    }
    halt("ThreeArray index out of bounds: ", i);
  }
}
\end{chapel}
\begin{chapelpost}
var ta = new ThreeArray();
ta(1) = 1;
ta(2) = 2;
ta(3) = 3;
for i in 1..3 do
  writeln(ta(i));
ta(4) = 4;
\end{chapelpost}
\begin{chapeloutput}
1
2
3
thisMethod.chpl:9: error: halt reached - ThreeArray index out of bounds: 4
\end{chapeloutput}
\end{chapelexample}

\section{The {\em these} Method}
\label{The_these_Method}
\index{these@\chpl{these}}
\index{classes!iterating}

An iterator method declared with the name \chpl{these} allows a class to be
``iterated over'' similarly to how a domain or array is iterated over.
Using a class in the context of a loop where
an \sntx{iteratable-expression} is expected has the semantics of calling
a method on the class named \chpl{these}.

\begin{chapelexample}{theseIterator.chpl}
In the following code, the \chpl{these} method is used to create a
class that acts like a simple array that can be iterated over and
contains three integers.
\begin{chapel}
class ThreeArray {
  var x1, x2, x3: int;
  iter these() var {
    yield x1;
    yield x2;
    yield x3;
  }
}
\end{chapel}
\begin{chapelpost}
var ta = new ThreeArray();
for (i, j) in (ta, 1..) do
  i = j;

for i in ta do
  writeln(i);
\end{chapelpost}
\begin{chapeloutput}
1
2
3
\end{chapeloutput}

\end{chapelexample}

\section{Common Operations}

\subsection{Class Assignment}
\label{Class_Assignment}
\index{classes!assignment}

Classes are assigned by reference.  After an assignment from one
variable of a class type to another, the variables reference the same
class instance.

\subsection{Implicit Class Conversions}
\label{Implicit_Class_Conversions}
\index{conversions!class}
\index{conversions!implicit!class}
\index{classes!implicit conversion}

An implicit conversion from class type \chpl{D} to
another class type \chpl{C} is allowed when \chpl{D} is a subclass
of \chpl{C}.
Such conversion does not change the value.

\section{Dynamic Memory Management}
\label{Dynamic_Memory_Management}
\label{Class_Delete}
\index{memory management}
\index{classes!delete}
\index{delete!classes}

Memory associated with class instances can be reclaimed with the \chpl{delete}
statement:

\begin{syntax}
delete-statement:
  `delete' expression ;
\end{syntax}

where the expression is a reference to the instance that will be reclaimed.

\begin{chapelexample}{delete.chpl}
The following example allocates a new object \chpl{c} of class type \chpl{C}
and then deletes it.
\begin{chapelpre}
class C {
  var i,j,k: int;
}
\end{chapelpre}
\begin{chapel}
var c = new C();
delete c;
\end{chapel}
\begin{chapelpost}
\end{chapelpost}
\begin{chapelexecopts}
--memLeaks
\end{chapelexecopts}
\begin{chapeloutput}

====================
Leaked Memory Report
==============================================================
Number of leaked allocations
           Total leaked memory (bytes)
                      Description of allocation
==============================================================
==============================================================
\end{chapeloutput}
\end{chapelexample}

\begin{openissue}
Chapel was originally specified without a \chpl{delete} keyword.  The intention
was that Chapel would be implemented with a distributed-memory garbage
collector.  This is a research challenge.  In order to focus elsewhere, the
design has been scaled back.  There is an expectation that Chapel will
eventually support an optional distributed-memory garbage collector as well as
a region-based memory management scheme similar to that used in the Titanium
language.  Support of \chpl{delete} will likely continue even as these optional
features become supported.
\end{openissue}
