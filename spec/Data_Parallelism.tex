\sekshun{Data Parallelism}
\label{Data_Parallelism}
\index{data parallelism}
\index{parallelism!data}

Chapel provides two explicit data-parallel constructs (the
forall-statement and the forall-expression) and several idioms that
support data parallelism implicitly (whole-array assignment, function
and operator promotion, reductions, and scans).

This chapter details data parallelism as follows:
\begin{itemize}
\item \rsec{Forall} describes the forall statement.
\item \rsec{Forall_Expressions} describes forall expressions
\item \rsec{Promotion} describes promotion.
\item \rsec{Reductions_and_Scans} describes reductions and scans
\item \rsec{data_parallel_knobs} describes the configuration constants for
controlling default data parallelism.
\end{itemize}       

\section{The Forall Statement}
\label{Forall}
\index{forall@\chpl{forall} (see also statements, forall)}
\index{loops!forall (see also statements, forall)}
\index{data parallelism!forall}
\index{statements!forall}

The forall statement is a concurrent variant of the for statement
described in~\rsec{The_For_Loop}.

\subsection{Syntax}
\label{forall_syntax}
\index{statements!forall!syntax}

The syntax of the forall statement is given by
\begin{syntax}
forall-statement:
  `forall' index-var-declaration `in' iteratable-expression `do' statement
  `forall' index-var-declaration `in' iteratable-expression block-statement
  `forall' iteratable-expression `do' statement
  `forall' iteratable-expression block-statement
  [ index-var-declaration `in' iteratable-expression ] statement
  [ iteratable-expression ] statement
\end{syntax}
As with the for statement, the indices may be omitted if they are
unnecessary and the \chpl{do} keyword may be omitted before a block
statement.  The square bracketed form is a syntactic convenience.

\subsection{Execution and Serializability}
\label{forall_semantics}
\index{statements!forall!semantics}

The forall statement evaluates the loop body once for each element
yielded by the \sntx{iteratable-expression}.  Each instance of the
forall loop's body may be executed concurrently with the others, but
this is not guaranteed.  In particular, the loop must be serializable.
Details regarding concurrency and iterator implementation are
described in~\ref{Parallel_Iterators}.

This differs from the semantics of the \chpl{coforall} loop, discussed
in~\rsec{Coforall}, where each iteration is guaranteed to run using
distinct tasks.  The \chpl{coforall} loop thus has potentially higher
overhead than a forall loop, but in cases where concurrency is
required for correctness, it is essential.

Control continues with the statement following the forall loop only
after every iteration has been completely evaluated.  At this point,
all data accesses within the body of the forall loop will be
guaranteed to be completed.

The following statements may not be lexically enclosed in forall
statements: break statements, and return statements.  Yield statement
may only be lexically enclosed in forall statements in parallel
iterators~(\rsec{Parallel_Iterators}).

\begin{chapelexample}{forallStmt.chpl}
In the code
\begin{chapelpre}
config const N = 5;
var a: [1..N] int;
var b = [i in 1..N] i;
\end{chapelpre}
\begin{chapel}
forall i in 1..N do
  a(i) = b(i);
\end{chapel}
the user has stated that the element-wise assignments can execute
concurrently.  This loop may be executed serially with a single task,
or by using a distinct task for every iteration, or by using a number
of tasks where each task executes a number of iterations.  This loop
can also be written as
\begin{chapel}
[i in 1..N] a(i) = b(i);
\end{chapel}
\begin{chapelpost}
writeln(a);
\end{chapelpost}
\begin{chapeloutput}
1 2 3 4 5
\end{chapeloutput}
\end{chapelexample}


\subsection{Zipper Iteration}
\label{forall_zipper}
\index{statements!forall!zipper iteration}

Zipper iteration has the same semantics as described
in~\rsec{Zipper_Iteration} and~\rsec{Parallel_Iterators} for parallel
iteration.

\section{The Forall Expression}
\label{Forall_Expressions}
\index{data parallelism!forall expressions}
\index{forall expressions (see also expressions, forall)}
\index{expressions!forall}

The forall expression is a concurrent variant of the for expression
described in~\rsec{For_Expressions}.

\subsection{Syntax}
\label{forall_expr_syntax}
\index{expressions!forall!syntax}

The syntax of a forall expression is given by
\begin{syntax}
forall-expression:
  `forall' index-var-declaration `in' iteratable-expression `do' expression
  `forall' iteratable-expression `do' expression
  [ index-var-declaration `in' iteratable-expression ] expression
  [ iteratable-expression ] expression
\end{syntax}
As with the for expression, the indices may be omitted if they are
unnecessary.  The \chpl{do} keyword is always required in the
keyword-based notation.  The bracketed form is a syntactic
convenience.

\subsection{Execution and Serializability}
\label{Forall_Expression_Execution_and_Serializability}
\index{expressions!forall!semantics}

The forall expression executes a forall loop (\rsec{Forall}),
evaluates the body expression on each iteration of the loop, and
returns the resulting values as a collection.  The size and shape of
that collection are determined by the iteratable-expression.

\begin{chapelexample}{forallExpr.chpl}
The code
\begin{chapel}
writeln(+ reduce [i in 1..10] i**2);
\end{chapel}
\begin{chapeloutput}
385
\end{chapeloutput}
applies a reduction to a forall-expression that evaluates the square
of the indices in the range \chpl{1..10}.
\end{chapelexample}

The forall expression follows the semantics of the forall statement as
described in~\ref{forall_semantics}.

\subsection{Zipper Iteration}
\index{expressions!forall!zipper iteration}
Forall expression also support zippered iteration semantics as
described in~\rsec{Zipper_Iteration} and~\rsec{Parallel_Iterators} for
parallel iteration.


\subsection{Filtering Predicates in Forall Expressions}
\label{Filtering_Predicates_Forall}
\index{expressions!forall!and conditional expressions}
\index{expressions!forall!filtering}

A filtering predicate is an if expression that is immediately enclosed
by a forall expression and does not have an
else clause.  Such an if expression filters the iterations of the
forall expression.  The iterations for which the condition does not
hold are not reflected in the result of the forall expression.

\begin{chapelexample}{forallFilter.chpl}
The following expression returns every other element starting with the
first:
\begin{chapelpre}
var s: [1..10] int = [i in 1..10] i;
var result =
\end{chapelpre}
\begin{chapel}
[i in 1..s.numElements] if i % 2 == 1 then s(i)
\end{chapel}
\begin{chapelpost}
;
writeln(result);
\end{chapelpost}
\begin{chapeloutput}
1 3 5 7 9
\end{chapeloutput}
\end{chapelexample}

\section{Promotion}
\label{Promotion}
\index{promotion}

A function that expects one or more scalar arguments but is called
with one or more arrays, domains, ranges, or iterators is promoted if
the element types of the arrays, the index types of the domains and/or
ranges, or the yielded types of the iterators can be resolved to the
type of the argument.  The rules of when an overloaded function can be
promoted are discussed in~\rsec{Function_Resolution}.

In addition to scalar functions, operators and casts are also
promoted.

\begin{chapelexample}{promotion.chpl}
Given the array
\begin{chapel}
var A: [1..5] int = [i in 1..5] i;
\end{chapel}
and the function
\begin{chapel}
proc square(x: int) return x**2;
\end{chapel}
then the call \chpl{square(A)} results in the promotion of
the \chpl{square} function over the values in the array \chpl{A}.  The
result is an iterator that returns the
values \chpl{1}, \chpl{4}, \chpl{9}, \chpl{16}, and \chpl{25}.
\begin{chapelnoprint}
for s in square(A) do writeln(s);
\end{chapelnoprint}
\begin{chapeloutput}
1
4
9
16
25
\end{chapeloutput}
\end{chapelexample}

%%
%% sungeun: 10/2011
%% I axed the following paragraph because comments from the peanut
%% gallery suggested it was confusing, too much implementation
%% details, etc.  One idea was to add something about defining
%% iterators to support promotion.  Not sure where that would
%% eventually go.
%%
%% If a promoted function returns a value, the promoted function becomes
%% an iterator that is controlled by a loop over the iterator (or array,
%% domain, or range) that it is promoted by.  If the function does not
%% return a value, the function is controlled by a loop over the iterator
%% that it is promoted by, but the promotion does not become an iterator.

Whole array operations are a form of promotion as applied to operators
rather than functions.


\subsection{Zipper Promotion}
\label{Zipper_Promotion}
\index{promotion!zipper iteration}

Promotion also supports zippered iteration semantics as described
in~\rsec{Zipper_Iteration} and~\rsec{Parallel_Iterators} for parallel
iteration.

Consider a function \chpl{f} with formal
arguments \chpl{s1}, \chpl{s2},~... that are promoted and formal
arguments \chpl{a1}, \chpl{a2},~... that are not promoted.  The call
\begin{chapel}
f(s1, s2, ..., a1, a2, ...)
\end{chapel}
is equivalent to
\begin{chapel}
[(e1, e2, ...) in (s1, s2, ...)] f(e1, e2, ..., a1, a2, ...)
\end{chapel}
The usual constraints of zipper iteration apply to zipper promotion so
the promoted actuals must have the same shape.

\begin{chapelexample}{zipper-promotion.chpl}
Given a function defined as
\begin{chapel}
proc foo(i: int, j: int) {
  return (i,j);
}
\end{chapel}
and a call to this function written
\begin{chapel}
writeln(foo(1..3, 4..6));
\end{chapel}
then the output is
\begin{chapelprintoutput}
(1, 4) (2, 5) (3, 6)
\end{chapelprintoutput}
\end{chapelexample}

\subsection{Whole Array Assignment}
\label{Whole_Array_Assignment}
\index{whole array assignment}
\index{arrays!assignment}
\index{assignment!whole array}

Whole array assignment is a considered a degenerate case of promotion
and is implicitly parallel.  The assignment statement
\begin{chapel}
LHS = RHS;
\end{chapel}
is equivalent to
\begin{chapel}
forall (e1,e2) in (LHS,RHS) do
  e1 = e2;
\end{chapel}

\subsection{Evaluation Order}
\label{Evaluation_Order}
\index{data parallelism!evaluation order}
The semantics of whole array assignment and promotion are different
from most array programming languages.  Specifically, the compiler
does not insert array temporaries for such operations if any of the
right-hand side array expressions alias the left-hand side expression.

%
% sungeun 4/8/2011
% Did not convert this one due to non-deterministic output
%
\begin{example}
If \chpl{A} is an array declared over the indices \chpl{1..5}, then
the following codes are not equivalent:
\begin{chapel}
A[2..4] = A[1..3] + A[3..5];
\end{chapel}
and
\begin{chapel}
var T = A[1..3] + A[3..5];
A[2..4] = T;
\end{chapel}
This follows because, in the former code, some of the new values that
are assigned to \chpl{A} may be read to compute the sum depending on
the number of tasks used to implement the data parallel statement.
\end{example}



\section{Reductions and Scans}
\label{Reductions_and_Scans}
\index{reductions}
\index{scans}
\index{data parallelism!reductions}
\index{data parallelism!scans}

Chapel provides reduction and scan expressions that apply operators to
aggregate expressions in stylized ways.  Reduction expressions
collapse the aggregate's values down to a summary value.  Scan
expressions compute an aggregate of results where each result value
stores the result of a reduction applied to all of the elements in the
aggregate up to that expression.  Chapel provides a number of predefined
reduction and scan operators, and also supports a mechanism for the
user to define additional reductions and
scans (Chapter~\ref{User_Defined_Reductions_and_Scans}).

\subsection{Reduction Expressions}
\label{reduce}
\index{reduction expressions}
\index{expressions!reduction}

A reduction expression applies a reduction operator to an aggregate
expression, collapsing the aggregate's dimensions down into a result
value (typically a scalar or summary expression that is independent of
the input aggregate's size).  For example, a sum reduction computes
the sum of all the elements in the input aggregate expression.

The syntax for a reduction expression is given by:
\begin{syntax}
reduce-expression:
  reduce-scan-operator `reduce' expression
  class-type `reduce' expression

reduce-scan-operator: one of
  + * && || & | ^ `min' `max' `minloc' `maxloc'
\end{syntax}

Chapel's predefined reduction operators are defined
by \sntx{reduce-scan-operator} above.  In order, they are: sum,
product, logical-and, logical-or, bitwise-and, bitwise-or,
bitwise-exclusive-or, minimum, maximum, minimum-with-location, and
maximum-with-location.  The minimum reduction returns the minimum
value as defined by the \verb@<@ operator.  The maximum reduction
returns the maximum value as defined by the \verb@>@ operator.  The
minimum-with-location reduction returns the lowest index position with
the minimum value (as defined by the \verb@<@ operator).  The
maximum-with-location reduction returns the lowest index position with
the maximum value (as defined by the \verb@>@ operator).

The expression on the right-hand side of the \chpl{reduce} keyword
can be of any type that can be iterated over, provided
the reduction operator can be applied to the values yielded
by the iteration. For example, the bitwise-and
operator can be applied to arrays of boolean or integral types to
compute the bitwise-and of all the values in the array.

For the minimum-with-location and maximum-with-location reductions,
the argument on the right-hand side of the \chpl{reduce} keyword
must be a 2-tuple. Its first component is the collection
of values for which the minimum/maximum value is to be computed.  The
second argument component is a collection of indices with the same size and
shape that provides names for the locations of the values in the first
component.  The reduction returns a tuple containing the
minimum/maximum value in the first argument component and the value
at the corresponding location in the second argument component.

\begin{chapelexample}{reduce-loc.chpl}
The first line below computes the smallest element in an array
\chpl{A} as well as its index, storing the results in \chpl{minA} and
\chpl{minALoc}, respectively.  It then computes the largest element in
a forall expression making calls to a function \chpl{foo()}, storing
the value and its number in \chpl{maxVal} and \chpl{maxValNum}.
\begin{chapelnoprint}
config const n = 10;
const D = [1..n];
var A: [D] int = [i in D] i % 7;
proc foo(x) return x % 7;
\end{chapelnoprint}
\begin{chapel}
var (minA, minALoc) = minloc reduce (A, A.domain); 
var (maxVal, maxValNum) = maxloc reduce ([i in 1..n] foo(i), 1..n);
\end{chapel}
\begin{chapelnoprint}
writeln((minA, minALoc));
writeln((maxVal, maxValNum));
\end{chapelnoprint}
\begin{chapeloutput}
(0, 7)
(6, 6)
\end{chapeloutput}
\end{chapelexample}

User-defined reductions are specified by preceding the
keyword \chpl{reduce} by the class type that implements the reduction
interface as described in~\rsec{User_Defined_Reductions_and_Scans}.

\subsection{Scan Expressions}
\label{scan}
\index{scan expressions}
\index{expressions!scan}

A scan expression applies a scan operator to an aggregate expression,
resulting in an aggregate expression of the same size and shape.  The
output values represent the result of the operator applied to all
elements up to and including the corresponding element in the input.

The syntax for a scan expression is given by:
\begin{syntax}
scan-expression:
  reduce-scan-operator `scan' expression
  class-type `scan' expression
\end{syntax}

The predefined scans are defined by \sntx{reduce-scan-operator}.  These
are identical to the predefined reductions and are described
in~\rsec{reduce}.

The expression on the right-hand side of the scan can be of any type
that can be iterated over and to which the operator can be applied.

%
% sungeun: 4/8/2011
% Did not convert this one yet due to warning about serializing scans
%
\begin{example}
Given an array
\begin{chapel}
var A: [1..3] int = 1;
\end{chapel}
that is initialized such that each element contains one, then the code
\begin{chapel}
writeln(+ scan A);
\end{chapel}
outputs the results of scanning the array with the sum operator.  The
output is
\begin{chapelprintoutput}
1 2 3
\end{chapelprintoutput}
\end{example}

User-defined scans are specified by preceding the keyword \chpl{scan}
by the class type that implements the scan interface as described
in Chapter~\ref{User_Defined_Reductions_and_Scans}.

\section{Configuration Constants for Default Data Parallelism}
\label{data_parallel_knobs}
\index{data parallelism!knobs for default data parallelism}
\index{data parallelism!configuration constants}

The following configuration constants are provided to control the
degree of data parallelism over ranges, default domains, and default
arrays:

\begin{center}
\begin{tabular}{|l|l|l|}
\hline
{\bf Config Const} & {\bf Type} & {\bf Default} \\
\hline
\chpl{dataParTasksPerLocale} & \chpl{int} & Number of cores per locale \\
\chpl{dataParIgnoreRunningTasks} & \chpl{bool} & \chpl{true} \\
\chpl{dataParMinGranularity} & \chpl{int} & \chpl{1} \\
\hline
\end{tabular}
\end{center}

The configuration constant \chpl{dataParTasksPerLocale} specifies the
number of tasks to use when executing a forall loop over a range,
default domain, or default array.  The actual number of tasks may be
fewer depending on the other two configuration constants.  A value of
zero results in using the default value.

The configuration constant \chpl{dataParIgnoreRunningTasks}, when
true, has no effect on the number of tasks to use to execute the
forall loop.  When false, the number of tasks per locale is decreased
by the number of tasks that are already running on the locale, with a
minimum value of one.

The configuration constant \chpl{dataParMinGranularity} specifies the
minimum number of iterations per task created.  The number of tasks is
decreased so that the number of iterations per task is never less than
the specified value.

For distributed domains and arrays that have these same configuration
constants (\eg, Block and Cyclic distributions), these same
global configuration constants are used to specify their default
behavior within each locale (\rsec{Block_Dist},~\rsec{Cyclic_Dist}).
