\sekshun{Domain Maps}
\label{Domain_Maps}
\index{domain maps}

A mapping from domain index values to locales is called a \emph{domain
map}.

\subsection{Domain Map Types}
\label{Domain_Map_Types}

Distribution types are defined by the type of a distribution class
that derives from BaseDist and is typically generic.  They are
distinct from the distribution class type.  Typically, the
distribution class type is only used on its own in defining the
distribution itself~\rsec{User_Defined_Domain_Maps}

Defining a distribution type involves specifying a distribution class
type and wrapping this in a distribution type \chpl{dist}.
\begin{example}
The code
\begin{chapel}
use BlockDist;
var MyBlockDist: dist(Block(rank=2));
\end{chapel}
defines a Block distribution called \chpl{MyBlockDist} with rank 2 and
a default index type that can be used to distribute 2-dimensional
arithmetic domains.  The Block distribution is described in more
detail in~\rsec{Block_Dist}.
\end{example}

\subsection{Domain Map Values}
\label{Domain_Map_Values}

Constructing a distribution involves calling the constructor of a
distribution class and defining a new distribution type \chpl{dist}.
\begin{example}
The code
\begin{chapel}
use BlockDist;
var MyBlockDist = new dist(new Block(rank=2, bbox=[1..n,1..n]));
\end{chapel}
constructs a Block distribution that partitions the index space
specified by \chpl{[1..n, 1..n]} over all of the locales.  The Block
distribution is described in more detail in~\rsec{Block_Dist}.
\end{example}

\subsection{Mapped Domains and Arrays}
\label{Mapped_Domains_and_Arrays}

\index{domains!distributed}
A domain for which a distribution is specified is referred to as a
{\em distributed domain}.

\begin{rationale}
Should this be? A domain supports a method, \chpl{locale}, that maps
index values in the domain to locales that correspond to the domain's
distribution.
\end{rationale}

The syntax to create a distributed domain type is the same as the
syntax to create a distributed domain value:
\begin{syntax}
distributed-domain-type:
  domain-type `distributed' distribution-expression

distributed-domain-expression:
  domain-expression `distributed' distribution-expression

distribution-expression:
  expression
\end{syntax}

\begin{example}
The code
\begin{chapel}
use BlockDist;
var MyBlockDist = new dist(new Block(rank=2, bbox=[1..n,1..n]));
var Dom: domain(2) distributed MyBlockDist =
           [1..n, 1..n] distributed MyBlockDist;
\end{chapel}
defines a new domain that is distributed by \chpl{MyBlockDist}.  Note
that, as usual, the type does not need to be specified if the variable
is initialized.
\end{example}

When defining a new distribution inline with the \chpl{distributed}
keyword, a syntactic sugar is supported in which the ``new dist new''
characters may be omitted.
\begin{example}
The code
\begin{chapel}
use BlockDist;
var D = [1..n, 1..n] distributed new dist(new Block(rank=2, bbox=[1..n,1..n]));
\end{chapel}
is equivalent to
\begin{chapel}
use BlockDist;
var D = [1..n, 1..n] distributed Block(rank=2, bbox=[1..n,1..n]);
\end{chapel}
\end{example}

Iteration over a distributed domain implicitly executes the controlled
task in the domain of the associated locale.  Similarly, when
iterating over the elements of an array defined over a distributed
domain, the controlled tasks are determined by the distribution of the
domain.  If there are conflicting distributions in product iterations,
the locale of a task is taken to be the first component in the
product.

\begin{example}
If \chpl{D} is a distributed domain, then in the code
\begin{chapel}
forall d in D {
  // body
}
\end{chapel}
the body of the loop is executed in the locale where the
index \chpl{d} maps to by the distribution of \chpl{D}.
\end{example}

\subsection{Undistributed Domains and Arrays}
\label{Undistributed_Domains_and_Arrays}

If a domain or an array does not have a distributed part, the domain
or array is not distributed and exists only on the locale on which it
is defined.
