\sekshun{Functions}
\label{Functions}
\index{functions}

This section defines functions.  Methods and iterators are functions
and most of this section applies to them as well.  They are defined
separately in~\rsec{Iterators} and~\rsec{Class_Methods}.


\subsection{Function Calls}
\label{Function_Calls}
\index{function calls}

The syntax to call a function is given by:
\begin{syntax}
call-expression:
  expression ( named-expression-list )
  expression [ named-expression-list ]
  parenthesesless-function-identifier

named-expression-list:
  named-expression
  named-expression , named-expression-list

named-expression:
  expression
  identifier = expression

parenthesesless-function-identifier:
  identifier
\end{syntax}


A \sntx{call-expression} is resolved to a particular function
according to the algorithm for function resolution described
in~\rsec{Function_Resolution}.

Functions can be called using either parentheses or brackets.  The
only difference in the call has to do with promotion and is discussed
in~\rsec{Promotion}.

Functions that are defined without parentheses are called without
parentheses as defined by scope resolution.  Functions without
parentheses are discussed in~\rsec{Functions_without_Parentheses}.

A \sntx{named-expression} is an expression that may be optionally
named.  The optional \sntx{identifier} represents a named actual
argument described in~\rsec{Named_Arguments}.


\subsection{Function Definitions}
\label{Function_Definitions}
\index{functions!syntax}

\index{def@\chpl{def}}
Functions are declared with the following syntax:
\begin{syntax}
function-declaration-statement:
  `def' function-name argument-list[OPT] var-param-type-clause[OPT] where-clause[OPT]
    function-body

function-name:
  identifier
  operator-name

operator-name: one of
  + - * / % ** ! == <= >= < > << >> & | ^ ~

argument-list:
  ( formals[OPT] )

formals:
  formal
  formal , formals

formal:
  formal-tag identifier formal-type[OPT] default-expression[OPT]
  formal-tag identifier formal-type[OPT] variable-argument-expression
  formal-tag tuple-grouped-identifier-list formal-type[OPT] default-expression[OPT]
  formal-tag tuple-grouped-identifier-list formal-type[OPT] variable-argument-expression

formal-type:
  : type-specifier
  : ? identifier[OPT]

default-expression:
  = expression

variable-argument-expression:
  ... expression
  ... ? identifier[OPT]
  ...

formal-tag: one of
  `in' `out' `inout' `param' `type'

var-param-type-clause:
  `var' return-type[OPT]
  `const' return-type[OPT]
  `param' return-type[OPT]
  `type'

return-type:
  : type-specifier

where-clause:
  `where' expression

function-body:
  block-statement
  return-statement
\end{syntax}

%% This should be in the order that the sections appear in this
%% chapter (if they appear in this chapter).

Functions do not require parentheses if they have no arguments.  Such
functions are described in~\rsec{Functions_without_Parentheses}.

Formal arguments can be grouped together using a tuple notation as
described in~\rsec{Formal_Argument_Declarations_in_a_Tuple}.

Default expressions allow for the omission of actual arguments at the
call site, resulting in the implicit passing of a default value.
Default values are discussed in~\rsec{Default_Values}.

The intents \chpl{in}, \chpl{out}, and \chpl{inout} are discussed
in~\rsec{Intents}.  The formal tags \chpl{param} and \chpl{type} make
a function generic and are discussed in~\rsec{Generics}.  If the
formal argument's type is elided, generic, or prefixed with a question
mark, the function is also generic and is discussed
in~\rsec{Generics}.

Functions can take a variable number of arguments.  Such functions are
discussed in~\rsec{Variable_Length_Argument_Lists}.

The optional \sntx{var-param-type-clause} defines a variable function,
discussed in~\rsec{Variable_Functions}, or a parameter function,
discussed in~\rsec{Parameter_Functions}, or a type function, discussed
in~\rsec{Type_Functions}.  By default, a function call cannot be
treated as an lvalue and is constant.  This may be explicitly
specified via the keyword~\chpl{const}.

Return types are optional and are discussed in~\rsec{Return_Types}.

The optional where clause is only applicable if the function is
generic.  It is discussed in~\rsec{Where_Expressions}.

Operator overloading is supported in Chapel on the operators listed
above under operator name.  Operator and function overloading is
discussed in~\rsec{Function_Overloading}.


\subsection{Functions without Parentheses}
\label{Functions_without_Parentheses}
\index{functions!functions without parentheses}

Functions do not require parentheses if they have empty argument
lists.  Functions declared without parentheses around empty argument
lists must be called without parentheses.

\begin{example}
Given the definitions
\begin{chapelpre}
% test_function_no_parens.chpl
\end{chapelpre}
\begin{chapel}
def foo { writeln("In foo"); }
def bar() { writeln("In bar"); }
\end{chapel}
\begin{chapelpost}
foo;
bar();
\end{chapelpost}
\begin{chapeloutput}
In foo
In bar
\end{chapeloutput}
the function \chpl{foo} can be called by writing \chpl{foo} and the
function \chpl{bar} can be called by writing \chpl{bar()}.  It is an
error to apply parentheses to \chpl{foo} or omit them from \chpl{bar}.
\end{example}


\subsection{Formal Arguments}
\label{Formal_Arguments}
\index{formal arguments}
\index{functions!formal arguments}

Chapel supports an intuitive formal argument passing mechanism.  An
argument is passed by value unless it is a class, array, or domain in
which case it is passed by reference.

Intents~(\rsec{Intents}) result in potential assignments to temporary
variables during a function call.  For example, passing an array by
intent \chpl{in}, a temporary array will be created.

\subsubsection{Named Arguments}
\label{Named_Arguments}
\index{named arguments}
\index{functions!named arguments}
\index{formal arguments!naming}

A formal argument can be named at the call site to explicitly map an
actual argument to a formal argument.

\begin{example}
In the code
\begin{chapelpre}
% test_named_args.chpl
\end{chapelpre}
\begin{chapel}
def foo(x: int, y: int) { writeln(x); writeln(y); }

foo(x=2, y=3);
foo(y=3, x=2);
\end{chapel}
\begin{chapeloutput}
2
3
2
3
\end{chapeloutput}
named argument passing is used to map the actual arguments to the
formal arguments.  The two function calls are equivalent.
\end{example}

Named arguments are sometimes necessary to disambiguate calls or
ignore arguments with default values.  For a function that has many
arguments, it is sometimes good practice to name the arguments at the
call-site for compiler-checked documentation.

\subsubsection{Default Values}
\label{Default_Values}
\index{default values}
\index{functions!default argument values}
\index{formal arguments!defaults}

Default values can be specified for a formal argument by appending the
assignment operator and a default expression the declaration of the
formal argument.  If the actual argument is omitted from the function
call, the default expression is evaluated when the function call is
made and the evaluated result is passed to the formal argument as if
it were passed from the call site.

\begin{example}
In the code
\begin{chapelpre}
% test_default_values.chpl
\end{chapelpre}
\begin{chapel}
def foo(x: int = 5, y: int = 7) { writeln(x); writeln(y); }

foo();
foo(7);
foo(y=5);
\end{chapel}
\begin{chapeloutput}
5
7
7
7
5
5
\end{chapeloutput}
default values are specified for the formal arguments \chpl{x}
and \chpl{y}.  The three calls to \chpl{foo} are equivalent to the
following three calls where the actual arguments are
explicit: \chpl{foo(5, 7)}, \chpl{foo(7, 7)}, and \chpl{foo(5, 5)}.
Note that named arguments are necessary to pass actual arguments to
formal arguments but use default values for arguments that are
specified earlier in the formal argument list.
\end{example}


\subsection{Intents}
\label{Intents}
\index{intents}
\index{functions!argument intents}

Intents allow the actual arguments to be copied to a formal argument
and also to be copied back.

\subsubsection{The Blank Intent}
\label{The_Blank_Intent}
\index{intents!blank}

If the intent is omitted, it is called a blank intent.  In such a
case, the value is copied in using the assignment operator.  Thus
classes are passed by reference and records are passed by value.
Arrays and domains are an exception because assignment does not apply
from the actual to the formal.  Instead, arrays and domains are passed
by reference.

With the exception of arrays, any argument that has blank intent
cannot be assigned within the function.

\subsubsection{The In Intent}
\label{The_In_Intent}
\index{in@\chpl{in}}
\index{intents!in@\chpl{in}}

If \chpl{in} is specified as the intent, the actual argument is copied
to the formal argument as usual, but it may also be assigned to within
the function.  This assignment is not reflected back at the call site.

If an array is passed to a formal argument that has \chpl{in} intent,
a copy of the array is made via assignment.  Changes to the elements
within the array are thus not reflected back at the call site.

\subsubsection{The Out Intent}
\label{The_Out_Intent}
\index{out@\chpl{out}}
\index{intents!out@\chpl{out}}

If \chpl{out} is specified as the intent, the actual argument is
ignored when the call is made, but after the call, the formal argument
is assigned to the actual argument at the call site.  The actual
argument must be a valid lvalue.  The formal argument can be assigned
to and read from within the function.

The formal argument cannot not be generic and is treated as a variable
declaration.

\subsubsection{The Inout Intent}
\label{The_Inout_Intent}
\index{inout@\chpl{inout}}
\index{intents!inout@\chpl{inout}}

If \chpl{inout} is specified as the intent, the actual argument is
both passed to the formal argument as if the \chpl{in} intent applied
and then copied back as if the \chpl{out} intent applied.  The formal
argument can be generic and takes its type from the actual argument.
The formal argument can be assigned to and read from within the
function.


\subsection{Variable Length Argument Lists}
\label{Variable_Length_Argument_Lists}
\index{functions!variable number of arguments}

Functions can be defined to take a variable number of arguments where
those arguments can have any intent or can be types.  A variable
number of parameters is not supported.  This allows the call site to
pass a different number of actual arguments.  There must be at least
one actual argument.

If the variable argument expression is an identifier prepended by a
question mark, the number of arguments is variable.  Alternatively,
the variable expression can evaluate to an integer parameter value
requiring the call site to pass that number of arguments to the
function.

In the function, the formal argument is a tuple of the actual
arguments.

\begin{example}
The code
\begin{chapelpre}
% test_varargs.chpl
\end{chapelpre}
\begin{chapel}
def mywriteln(x ...?k) {
  for param i in 1..k do
    writeln(x(i));
}
\end{chapel}
\begin{chapelpost}
mywriteln("hi", "there");
mywriteln(1, 2.0, 3, 4.0);
\end{chapelpost}
\begin{chapeloutput}
hi
there
1
2.0
3
4.0
\end{chapeloutput}
defines a generic function called \chpl{mywriteln} that takes a
variable number of arguments of any type and then writes them out on
separate lines.  The parameter for-loop~(\rsec{Parameter_For_Loops})
is unrolled by the compiler so that \chpl{i} is a parameter, rather
than a variable.  This needs to be a parameter for-loop because the
expression \chpl{x(i)} will have a different type on each iteration.
The type of \chpl{x} can be specified in the formal argument list to
ensure that the actuals all have the same type.
\end{example}

\begin{example}
Either or both the number of variable arguments and their types can be
specified.  For example, a basic function to sum the values of three
integers can be wrtten as
\begin{chapelpre}
% test_varargs_with_type.chpl
\end{chapelpre}
\begin{chapel}
def sum(x: int...3) return x(1) + x(2) + x(3);
\end{chapel}
\begin{chapelpost}
writeln(sum(1, 2, 3));
writeln(sum(-1, -2, -3));
\end{chapelpost}
\begin{chapeloutput}
6
-6
\end{chapeloutput}
Specifying the type is useful if it is important that each argument
have the same type.  Specifying the number is useful in, for example,
defining a method on a class that is instantiated over a rank
parameter.
\end{example}

\begin{example}
The function
\begin{chapelpre}
% test_varargs_returns_tuples.chpl
\end{chapelpre}
\begin{chapel}
def tuple(x ...) return x;
\end{chapel}
\begin{chapelpost}
writeln(tuple(1));
writeln(tuple("hi", "there"));
writeln(tuple(tuple(1, 2), tuple(3, 4)));
\end{chapelpost}
\begin{chapeloutput}
(1)
(hi, there)
((1, 2), (3, 4))
\end{chapeloutput}
creates a generic function that returns tuples.  When passed two or
more actuals in a call, it is equivalent to building a tuple so the
expressions \chpl{tuple(1, 2)} and \chpl{(1, 2)} are equivalent.  When
passed one actual, it builds a 1-tuple which is different than the
evaluation of the parenthesized expression.  Thus the
expressions \chpl{tuple(1)} and \chpl{(1)} are not equivalent.
\end{example}


\subsection{Variable Functions}
\label{Variable_Functions}
\index{functions!variable functions}
\index{functions!as lvalues}

A variable function is a function that can be assigned a value.  Note
that a variable function does not return a reference.  That is, the
reference cannot be captured.

A variable function is specified by following the argument list with
the \chpl{var} keyword.  A variable function must return an lvalue.

When a variable function is called on the left-hand side of an
assignment statement or in the context of a call to a formal argument
by out or inout intent, the lvalue that is returned by the function is
assigned a value.

Variable functions support an implicit argument \chpl{setter} of type
bool that is a compile-time constant (and can thus be folded).  If the
variable function is called in a context such that the returned lvalue
is assigned a value, the argument \chpl{setter} is \chpl{true};
otherwise it is \chpl{false}.  This argument is useful for controlling
different behavior depending on the call site.

\begin{example}
The following code creates a function that can be interpreted as a
simple two-element array where the elements are actually global
variables:
\begin{chapelpre}
% test_variable_functions.chpl
% This test spans the next 3 chapel code segments
\end{chapelpre}
\begin{chapel}
var x, y = 0;

def A(i: int) var {
  if i < 0 || i > 1 then
    halt("array access out of bounds");
  if i == 0 then
    return x;
  else
    return y;
}
\end{chapel}
This function can be assigned to in order to write to the ``elements''
of the array as in
\begin{chapel}
A(0) = 1;
A(1) = 2;
\end{chapel}
It can be called as an expression to access the ``elements'' as in
\begin{chapel}
writeln(A(0) + A(1));
\end{chapel}
\begin{chapeloutput}
3
\end{chapeloutput}
This code outputs the number \chpl{3}.

\index{setter}
\index{functions!setter argument}
The implicit \chpl{setter} argument can be used to ensure, for
example, that the second element in the pseudo-array is only assigned
a value if the first argument is positive.  To do this, add the
following:
\begin{chapelpre}
% test_setter.chpl
var x, y = 0;
def A(i: int) var { // } to fool latex
  if i < 0 || i > 1 then
    halt("array access out of bounds");
\end{chapelpre}
\begin{chapel}
if setter && i == 1 && x <= 0 then
  halt("cannot assign value to A(1) if A(0) <= 0");
\end{chapel}
\begin{chapelpost}
// { to fool latex
  if i == 0 then
    return x;
  else
    return y;
}
A(1) = 1;
\end{chapelpost}
\begin{chapeloutput}
test\_setter.chpl:6: error: halt reached - cannot assign value to A(1) if A(0) <= 0
\end{chapeloutput}
\end{example}


\subsection{Parameter Functions}
\label{Parameter_Functions}
\index{functions!as parameters}

A parameter function is a function that returns a parameter
expression.  It is specified by following the function's argument list
by the keyword \chpl{param}.  It is often, but not necessarily,
generic.

It is a compile-time error if a parameter function does not return a
parameter expression.  The result of a parameter function is computed
during compilation and the result is inlined at the call site.

\begin{example}
In the code
\begin{chapelpre}
% test_param_functions.chpl
\end{chapelpre}
\begin{chapel}
def sumOfSquares(param a: int, param b: int) param
  return a**2 + b**2;

var x: sumOfSquares(2, 3)*int;
\end{chapel}
\begin{chapelpost}
writeln(x);
\end{chapelpost}
\begin{chapeloutput}
(0, 0, 0, 0, 0, 0, 0, 0, 0, 0, 0, 0, 0)
\end{chapeloutput}
the function \chpl{sumOfSquares} is a parameter function that takes
two parameters as arguments.  Calls to this function can be used in
places where a parameter expression is required.  In this example, the
call is used in the declaration of a homogeneous tuple and so is
required to be a parameter.
\end{example}.

Parameter functions may not contain control flow that is not resolved
at compile-time.  This includes loops other than the parameter for
loop~\rsec{Parameter_For_Loops} and conditionals with a conditional
expressions that is not a parameter.


\subsection{Type Functions}
\label{Type_Functions}
\index{functions!as types}

A type function is a function that returns a type.  It is specified by
following the function's argument list by the keyword \chpl{type}.  It
is often, but not necessarily, generic.

It is a compile-time error if a type function does not return a type.
The result of a type function is computed during compilation.

As with parameter functions, type functions may not contain control
flow that is not resolved at compile-time.  This includes loops other
than the parameter for loop~\rsec{Parameter_For_Loops} and
conditionals with a conditional expressions that is not a parameter.

\begin{example}
In the code
\begin{chapelpre}
% test_type_functions.chpl
\end{chapelpre}
\begin{chapel}
def myType(x) type {
  if numBits(x.type) <= 32 then return int;
  else return int(64);
}
\end{chapel}
\begin{chapelpost}
var a = 4,
    b = 4.0;
var at: myType(a),
    bt: myType(b);
writeln((numBits(at.type), numBits(bt.type)));
\end{chapelpost}
\begin{chapeloutput}
(32, 64)
\end{chapeloutput}
the function \chpl{myType} is a type function that takes a single
argument \chpl{x} and returns \chpl{int} if the number of bits used to
represent \chpl{x} is less than or equal to 32, otherwise it
returns \chpl{int(64)}.  The function \chpl{numBits} is a param
function defined in the Standard Modules.
\end{example}


\subsection{The Return Statement}
\label{The_Return_Statement}
\index{return@\chpl{return}}

The return statement can only appear in a function.  It exits that
function, returning control to the point at which that function was
called.  It can optionally return a value.  The syntax of the return
statement is given by
\begin{syntax}
return-statement:
  `return' expression[OPT] ;
\end{syntax}

\begin{example}
The following code defines a function that returns the sum of three
integers:
\begin{chapelpre}
% test_return.chpl
\end{chapelpre}
\begin{chapel}
def sum(i1: int, i2: int, i3: int)
  return i1 + i2 + i3;
\end{chapel}
\begin{chapelpost}
writeln(sum(1, 2, 3));
\end{chapelpost}
\begin{chapeloutput}
6
\end{chapeloutput}
\end{example}


\subsection{Return Types}
\label{Return_Types}
\index{return@\chpl{return}!types}
\index{functions!return@\chpl{return} types}

A function can optionally return a value.  If the function does not
return a value, then no return type can be specified.  If the function
does return a value, the return type is optional.

\subsubsection{Explicit Return Types}
\label{Explicit_Return_Types}

If a return type is specified, the values that the function returns
via return statements must be assignable to a value of the return
type.  For variable functions~(\rsec{Variable_Functions}), the return
type must match the type returned in all of the return statements
exactly.

\subsubsection{Implicit Return Types}
\label{Implicit_Return_Types}
\index{type inference!of return types}

If a return type is not specified, it will be inferred from the return
statements.  Given the types that are returned by the different
statements, if exactly one of those types can be a target, via
implicit conversions, of every other type, then that is the inferred
return type.  Otherwise, it is an error.  For variable
functions~(\rsec{Variable_Functions}), every return statement must
return the same exact type and it becomes the inferred type.


\subsection{Function Overloading}
\label{Function_Overloading}
\index{functions!overloading}
\index{operators!overloading}

Functions that have the same name but different argument lists are
called overloaded functions.  Function calls to overloaded functions
are resolved according to the algorithm in~\rsec{Function_Resolution}.

Operator overloading is achieved by defining a function with a name
specified by that operator.  The operators that may be overloaded are
listed in the following table:

\begin{center}
\begin{tabular}{|l|l|}
\hline
{\bf arity} & {\bf operators} \\
\hline
unary & \verb@+ - ! ~@ \\
binary & \verb@+ - * / % ** ! == <= >= < > << >> & | ^ @ \\
\hline
\end{tabular}
\end{center}

The arity and precedence of the operator must be maintained when it is
overloaded.  Operator resolution follows the same algorithm as
function resolution.


\subsection{Nested Functions}
\label{Nested_Functions}
\index{functions!nested}

A function defined in another function is called a nested function.
Nesting of functions may be done to arbitrary degrees, i.e., a
function can be nested in a nested function.

Nested functions are only visible to function calls within the scope
in which they are defined.

\subsubsection{Accessing Outer Variables}
\label{Accessing_Outer_Variables}

Nested functions may refer to variables defined in the function in
which they are nested.


\subsection{First-Class Functions}
\label{First_Class_Functions}
\index{functions!first-class}
\index{functions!capture}

Functions defined with parentheses may be captured as values by referring to them by name without parentheses.  Once captured, these values may be passed around as other value types.

\begin{example}
\begin{chapelpre}
% test_first_class_functions.chpl
\end{chapelpre}
\begin{chapel}
def myfunc(x:int) { return x + 1; }

var f = myfunc;
writeln(f(3)); // outputs: 4
\end{chapel}
\begin{chapeloutput}
4
\end{chapeloutput}
\end{example}

To be captured, a function must not be any of the following:

\begin{itemize}
\item
  A generic function - all captured functions must be fully-qualified with no generic arguments
\item
  A function with special return types (type, param)
\item
  An iterator
\item
  The method of an object
\item
  An operator
\item
  An overloaded function
\end{itemize}

\begin{rationale}
Generic functions are outside of the scope of the current first-class function work.  Functions with compile-time return types like type and param require the ability to have param classes.  Param classes are not currently part of Chapel.  Iterators require a new type of capture, one that works similarly to the current implementation but respects the yielding that occurs inside an interator.  Method capture requires the currying of the object as the first argument to the first-class function.  Operators and overloaded functions require a type-based multiple dispatch mechanism.
\end{rationale}


\subsection{Lambda Functions}
\label{Lambda_Functions}
\index{lambda functions}
\index{functions!lambdas}

\begin{syntax}
lambda-declaration-expression:
  `lambda' argument-list
    return-type[OPT] function-body
\end{syntax}


Anonymous first-class functions, those not formally defined as named functions, are  available using the lambda keyword.

\begin{example}
\begin{chapelpre}
% test_lambda.chpl
\end{chapelpre}
\begin{chapel}
var f = lambda(x:int, y:int) { return x + y; };
writeln(f(1,2)); // outputs: 3
\end{chapel}
\begin{chapeloutput}
3
\end{chapeloutput}
\end{example}

\index{functions!capture variables}
\index{functions!closures}
\index{closures}

Lambdas can also \textit{capture} variables that are defined outside of the lambda by referring to them in the body of the lambda.  These form a \textit{closure}, a combination of a function and an associated execution environment.  This closure captures the variables in such a way that modifying them modifies the original variables (this is sometimes called capturing the variables by reference).  

\begin{example}
For example the following is acceptable:
\begin{chapelpre}
% test_lambda_capture.chpl
\end{chapelpre}
\begin{chapel}
def myfunc() {
  var x = 3;
  var f = lambda() { x = 4; };
  f();
  return x;
}
writeln(myfunc()); // outputs: 4
\end{chapel}
\begin{chapeloutput}
4
\end{chapeloutput}
\end{example}


\subsection{Function Type Signatures}
\label{Function_Type_Signatures}
\index{functions!type signatures}

\begin{syntax}
function-type:
  `func'( type-args[OPT] )

type-args:
  function-return-type
  type-specifier, function-return-type
  type-specifier, type-args

function-return-type:
  type-specifier
  
\end{syntax}

\begin{example}
\begin{chapelpre}
% test_function_type_signatures.chpl
\end{chapelpre}
\begin{chapel}
var f : func(); // A function with no arguments, returning void
var g : func(int); // A function with no arguments, returning int
var h : func(bool, int); // A function with one bool argument, returning int
\end{chapel}
\begin{chapelpost}
def f1() {
  writeln("In f1");
}
def g1() {
  return -1;
}
def h1(b: bool) {
  if b then return -1;
  else return 1;
}
f = f1;
g = g1;
h = h1;
f();
writeln(g());
writeln(h(true));
writeln(h(false));
\end{chapelpost}
\begin{chapeloutput}
In f1
-1
-1
1
\end{chapeloutput}
\end{example}

Where func is a keyword function which takes any number of argument types to the function and finally a type that represents the return type of the function.  Functions which have no arguments drop them from the call to func and supply the return type alone.  A function with no arguments and void return type may use the shortcut \chpl{func()}.
                          


\subsection{Function Resolution}
\label{Function_Resolution}
\index{function resolution}
\index{functions!resolution}

Given a function call, the function that the call resolves to is
determined according to the following algorithm:
\begin{itemize}
\item
Identify the set of visible functions for the function call.  A
\emph{visible function} is any function that satisfies the criteria
in~\rsec{Determining_Visible_Functions}.  If no visible function can
be found, the compiler will issue an error stating that the call
cannot be resolved.
\item
From the set of visible functions for the function call, determine the
set of candidate functions for the function call.  A \emph{candidate
function} is any function that satisfies the criteria
in~\rsec{Determining_Candidate_Functions}.  If no candidate function
can be found, the compiler will issue an error stating that the call
cannot be resolved.  If exactly one candidate function is found, this
is determined to be the function.
\item
From the set of candidate functions, the most specific function is
determined.  The most specific function is a candidate function that
is \emph{more specific} than every other candidate function as defined
in~\rsec{Determining_More_Specific_Functions}.  If there is no
function that is more specific than every other candidate function,
the compiler will issue an error stating that the call is ambiguous.
\end{itemize}.

\subsubsection{Determining Visible Functions}
\label{Determining_Visible_Functions}
\index{functions!visible}

Given a function call, a function is determined to be a \emph{visible
function} if the name of the function is the same as the name of the
function call and the function is defined in the same scope as the
function call or a lexical outer scope of the function call, or if the
function is defined in a module that is used from the same scope as
the function call or a lexical outer scope fo the function call.
Function visibility in generic functions is discussed
in~\rsec{Function_Visibility_in_Generic_Functions}.

\subsubsection{Determining Candidate Functions}
\label{Determining_Candidate_Functions}
\index{functions!candidates}

Given a function call, a function is determined to be
a \emph{candidate function} if there is a \emph{valid mapping} from
the function call to the function and each actual argument is mapped
to a formal argument that is a \emph{legal argument mapping}.

\paragraph{Valid Mapping}

The following algorithm determines a valid mapping from a function
call to a function if one exists:
\begin{itemize}
\item
Each actual argument that is passed by name is matched to the formal
argument with that name.  If there is no formal argument with that
name, there is no valid mapping.
\item
The remaining actual arguments are mapped in order to the remaining
formal arguments in order.  If there are more actual arguments then
formal arguments, there is no valid mapping.  If any formal argument
that is not mapped to by an actual argument does not have a default
value, there is no valid mapping.
\item
The valid mapping is the mapping of actual arguments to formal
arguments plus default values to formal arguments that are not mapped
to by actual arguments.
\end{itemize}

\paragraph{Legal Argument Mapping}

An actual argument of type $T_A$ can be mapped to a formal argument of
type $T_F$ if any of the following conditions hold:
\begin{itemize}
\item $T_A$ and $T_F$ are the same type.
\item There is an implicit coercion from $T_A$ to $T_F$.
\item $T_A$ is derived from $T_F$.
\item $T_A$ is scalar promotable to $T_F$.
\end{itemize}

\subsubsection{Determining More Specific Functions}
\label{Determining_More_Specific_Functions}
\index{functions!most specific}

Given two functions $F_1$ and $F_2$, the more specific function is
determined by the following steps:
\begin{itemize}
\item If $F_1$ does not require promotion and $F_2$ does require promotion, then $F_1$ is more specific.
\item If $F_2$ does not require promotion and $F_1$ does require promotion, then $F_2$ is more specific.
\item
If at least one of the legal argument mappings to $F_1$ is a {\em more
specific argument mapping} than the corresponding legal argument
mapping to $F_2$ and none of the legal argument mappings to $F_2$ is a
more specific argument mapping than the corresponding legal argument
mapping to $F_1$, then $F_1$ is more specific.
\item
If at least one of the legal argument mappings to $F_2$ is a {\em more
specific argument mapping} than the corresponding legal argument
mapping to $F_1$ and none of the legal argument mappings to $F_1$ is a
more specific argument mapping than the corresponding legal argument
mapping to $F_2$, then $F_2$ is more specific.
\item If $F_1$ shadows $F_2$, then $F_1$ is more specific.
\item If $F_2$ shadows $F_1$, then $F_2$ is more specific.
\item If $F_1$ has a where clause and $F_2$ does not have a where clause, then $F_1$ is more specific.
\item If $F_2$ has a where clause and $F_1$ does not have a where clause, then $F_2$ is more specific.
\item Otherwise neither function is more specific.
\end{itemize}

Given an argument mapping, $M_1$, from an actual argument, $A$, of
type $T_A$ to a formal argument, $F1$, of type $T_{F1}$ and an
argument mapping, $M_2$, from the same actual argument to a formal
argument, $F2$, of type $T_{F2}$, the more specific argument mapping
is determined by the following steps:
\begin{itemize}
\item
 If $T_{F1}$ and $T_{F2}$ are the same type, $F1$ is an instantiated
 parameter, and $F2$ is not an instantiated parameter, $M_1$ is more
 specific.
\item
 If $T_{F1}$ and $T_{F2}$ are the same type, $F2$ is an instantiated
 parameter, and $F1$ is not an instantiated parameter, $M_2$ is more
 specific.
\item
 If $M_1$ does not require scalar promotion and $M_2$ requires scalar
 promotion, $M_1$ is more specific.
\item
 If $M_1$ requires scalar promotion and $M_2$ does not require scalar
 promotion, $M_2$ is more specific.
\item
 If $T_{F1}$ and $T_{F2}$ are the same type, $F1$ is generic, and $F2$
 is not generic, $M_1$ is more specific.
\item
 If $T_{F1}$ and $T_{F2}$ are the same type, $F2$ is generic, and $F1$
 is not generic, $M_2$ is more specific.
\item
 If $F1$ is not generic over all types and $F2$ is generic over all
 types, $M_1$ is more specific.
\item
 If $F1$ is generic over all types and $F2$ is not generic over all
 types, $M_2$ is more specific.
\item
 If $T_A$ and $T_{F1}$ are the same type and $T_A$ and $T_{F2}$ are
 not the same type, $M_1$ is more specific.
\item
 If $T_A$ and $T_{F1}$ are not the same type and $T_A$ and $T_{F2}$
 are the same type, $M_2$ is more specific.
\item
 If $T_{F1}$ is derived from $T_{F2}$, then $M_1$ is more specific.
\item
 If $T_{F2}$ is derived from $T_{F1}$, then $M_2$ is more specific.
\item
 If there is an implicit coercion from $T_{F1}$ to $T_{F2}$, then
 $M_1$ is more specific.
\item
 If there is an implicit coercion from $T_{F2}$ to $T_{F1}$, then
 $M_2$ is more specific.
\item
 If $T_{F1}$ is any \chpl{int} type and $T_{F2}$ is any \chpl{uint}
 type, $M_1$ is more specific.
\item
 If $T_{F2}$ is any \chpl{int} type and $T_{F1}$ is any \chpl{uint}
 type, $M_2$ is more specific.
\item
 Otherwise neither mapping is more specific.
\end{itemize}
