\sekshun{Input and Output}
\label{Input_and_Output}
% got one right below: \index{input and output}
\index{input/output}

Input/output (I/O) facilities in Chapel include
the types \chpl{file} and \chpl{channel},
the constants \chpl{stdin}, \chpl{stdout} and \chpl{stderr},
the functions \chpl{open}, \chpl{close}, \chpl{reader}, \chpl{writer},
\chpl{read} and \chpl{write},
and others.

\index{input/output!file@\chpl{file}}
\index{input/output!channel@\chpl{channel}}
A \emph{file} in Chapel identifies a file in the underlying operating system.
% TODO: define the typical sequence as "open a file" -> "create a channel"
%
Reads and writes to a file are done via one or more \emph{channels}
associated with the file.
Each channel provides sequential access to its file, optionally
starting at an offset. Channels operate independently,
thus enabling concurrent I/O without contending for locks.
The behavior in the case of concurrent operations
on overlapping channels for the same file is undefined, however.
%
Explicit synchronization operations can be used
to make the data written to a channel available to other channels
and to commit the data to the file device for persistence.

\index{input/output!I/O style}
Reading and writing of Chapel's basic types is regulated by
an applicable \emph{I/O style}.
In particular, the I/O style controls whether binary or text I/O
should be performed. For binary I/O it specifies, for example, byte order
and string encoding. For text I/O it specifies string representation,
the base, field width and precision for numeric types, and so on.
Each channel has an associated I/O style. It applies to
all read/write operations on that channel, except when the program
specifies explicitly an I/O style for a particular read or write.

I/O facilities in Chapel also include the ability to write data
to strings and to define dynamically the output format or actions
to be taken when writing values of a given type.

I/O facilities are presented as follows:

\begin{itemize}

\item the \chpl{file} type and file operations
      \rsec{IO_files}

\item the \chpl{channel} type and channel operations
      \rsec{IO_channels}

\item I/O style details
      \rsec{IO_io_style}

\item the \chpl{stdin}, \chpl{stdout} and \chpl{stderr} channels
      \rsec{IO_std_channels}

\item the supporting data types for I/O
      \rsec{IO_data_types}

\item error handling for I/O functions
      \rsec{IO_error_handling}

\item writing and reading operations and supporting data types
      \rsec{IO_writing_reading}

\end{itemize}


% TODO: MF suggests adding a section with a further example about
% how to use a channel or a file.


\section{Files}
\label{IO_files}
\index{input/output!file@\chpl{file}}

A \chpl{file} value (\rsec{IO_file_type})
is produced when opening a file (\rsec{IO_open_file})
and represents a file in the underlying operating system.
%
Once a file is open, it is necessary to create associated channel(s)
(\rsec{IO_channel_creation}) to write to and/or read from the OS file.
%
Explicit synchronization (\rsec{IO_file_synchronization}) can be used to ensure
that file data is committed to the file's OS device.
%
To release any resources associated with a file,
its channels then the file must be closed
(\rsec{IO_closing_channels}, \rsec{IO_closing_files}).


\subsection{The {\em file} Type and Values}
\label{IO_file_type}
\index{input/output!file@\chpl{file}}
\index{input/output!types!file@\chpl{file}}

The \chpl{file} type is implementation-defined.
A value of the \chpl{file} type refers to the state that is used
by the implementation to identify and interact with the OS file.

When a \chpl{file} formal argument has blank intent, the
actual is copied to the formal upon a function call and
the formal cannot be assigned within the function.

\index{input/output!file@\chpl{file}!default value}
The default value of the \chpl{file} type does not represent any OS file.
It is illegal to perform any I/O operations on the default value.


\subsection{Functions for Opening Files}
\label{IO_open_file}
\index{input/output!open, opentmp, openmem@\chpl{open}, \chpl{opentmp}, \chpl{openmem}}

A file can be opened using the following functions.

\begin{protohead}
proc open(out error:syserr, path:string, mode:iomode,
          hints:iohints=IOHINT_NONE, style:iostyle = new iostyle()):file;
proc open(path:string, mode:iomode,
          hints:iohints=IOHINT_NONE, style:iostyle = new iostyle()):file;
\end{protohead}
\begin{protobody}
Opens a file specified by $path$, with
the access mode $mode$ (\rsec{IO_iomode_type})
and the hints $hints$ (\rsec{IO_iohints_type}).
The $style$ argument defines the I/O style (\rsec{IO_io_style})
for use by default when creating channels associated with this file
(\rsec{IO_channel_creation}).
The \chpl{error} argument is explained in \rsec{IO_error_handling}.
\end{protobody}

\begin{protohead}
proc opentmp(out error:syserr,
             hints:iohints=IOHINT_NONE, style:iostyle = new iostyle()):file;
proc opentmp(hints:iohints=IOHINT_NONE, style:iostyle = new iostyle()):file;
\end{protohead}
\begin{protobody}
% TODO: for writing only?
Similar to \chpl{open}, but the created file is temporary:
it is a new file, is created in an OS-dependent temporary directory (e.g. \chpl{/tmp})
and is deleted upon closing.
As with a file opened with the \chpl{iomode.cwr} access mode,
both writing to and reading from this file is allowed.
\end{protobody}

\begin{protohead}
proc openmem(out error:syserr, style:iostyle = new iostyle());
proc openmem(style:iostyle = new iostyle()):file;
\end{protohead}
\begin{protobody}
% TODO: for writing only?
Similar to \chpl{opentmp}, but
the created file is backed up by a buffer in memory, not by a disk.
\end{protobody}


\subsection{Synchronization of File Data}
\label{IO_file_synchronization}

The following synchronization operation is available.

\index{input/output!fsync@\chpl{fsync}}
\index{predefined functions!fsync@\chpl{fsync}}
\begin{protohead}
proc file.fsync(out error:syserr);
proc file.fsync();
\end{protohead}
\begin{protobody}
Commits file data to the device associated with this file.
Data written to the file by a channel will be committed
only if the channel has been closed or flushed.
The \chpl{error} argument is explained in \rsec{IO_error_handling}.
\end{protobody}
\begin{craychapel}
\chpl{fsync} in Chapel translates to the \chpl{fsync} system call.
\end{craychapel}


\subsection{Functions for Closing Files}
\label{IO_closing_files}

In order to free the resources allocated for a file, it
must be closed using one of the following methods:

\index{input/output!close (file)@\chpl{close} (file)}
\begin{protohead}
proc file.close(out error:syserr);
proc file.close();
\end{protohead}
\begin{protobody}
Close a file.
The \chpl{error} argument is explained in \rsec{IO_error_handling}.
\end{protobody}

%TODO: can one be closed multiple times?
It is an error to perform any I/O operations on a file
that has been closed.
It is an error to close a file when it has channels that
have not been closed.

Closing a file does not guarantee persistence of the
performed updates, if any.
\chpl{fsync} (\rsec{IO_file_synchronization}) should be used for that purpose
prior to closing the file.

\begin{future}
In the future we plan to implement reference counting for files
and channels. Each file and channel will be closed automatically
when no references remain to it. For example, if only a local
variable refers to a channel, the channel will be closed
when that variable goes out of scope.
% any more here?
The ability for the program to close a file or a channel
explicitly will remain available.
\end{future}


\section{Channels}
\label{IO_channels}
% got one right below: \index{input/output!channel@\chpl{channel}}


\subsection{The {\em channel} Type and Values}
\label{IO_channel_type}
\index{input/output!channel@\chpl{channel}}
\index{input/output!types!channel@\chpl{channel}}

The \chpl{channel} type is implementation-defined.
A value of the \chpl{channel} type refers to the state that is used
to implement the channel operations.

When a \chpl{channel} formal has blank intent, the
actual is copied to the formal upon a function call and
the formal cannot be assigned within the function.

\index{input/output!channel@\chpl{channel}!default value}
The default value of the \chpl{channel} type is not associated
with any file and so cannot be used to perform I/O.

The \chpl{channel} type is generic. It has the following arguments,
none of which have default values:

\begin{itemize}

\item \chpl{writing} is a boolean indicating whether the channels of this type
      support writing (when \chpl{true}) or reading (when \chpl{false}).

\item \chpl{kind} is an enum \chpl{iokind} that allows narrowing
      this channel's I/O style for more efficient binary I/O,
      as described in (\rsec{IO_iokind_type}).

\item \chpl{locking} is a boolean indicating whether it is
      safe to use this channel concurrently (when \chpl{true}).

\end{itemize}


\subsection{Functions for Channel Creation}
\label{IO_channel_creation}

\index{input/output!writer@\chpl{writer}}
\begin{protohead}
proc file.writer(out error:syserr, param kind=iokind.dynamic, param locking=true,
                 start:int(64) = 0, end:int(64) = max(int(64)),
                 hints:iohints = IOHINT_NONE,
                 style:iostyle = this._style): channel(true,kind,locking);
proc file.writer(param kind=iokind.dynamic, param locking=true,
                 start:int(64) = 0, end:int(64) = max(int(64)),
                 hints:iohints = IOHINT_NONE,
                 style:iostyle = this._style): channel(true,kind,locking);
\end{protohead}
\begin{protobody}
Creates a channel for writing to the file.

The arguments are as follows:

\begin{itemize}

\item \chpl{error} is explained in \rsec{IO_error_handling}.

\item \chpl{kind} and \chpl{locking} determine the corresponding parameters of
      the \chpl{channel} type that is returned (see \rsec{IO_channel_type})

\item \chpl{start} and \chpl{end} define the region of the file that
      the channel will write to. 
      These are byte offsets; the beginning of the file is at the offset 0.
      The defaults for these arguments enable
      the channel to access the entire file.

      A channel will never write beyond its maximum end position.
      It will extend the file only as necessary to store data
      written to the channel. In other words, specifying end here
      does not impact the file size directly; it impacts only the section of the file
      that this channel can write to. After all channels to a file
      are closed, that file will have a size equal to the last
      position written to by any channel.

\item \chpl{hints} provides hints about the I/O that this channel
      will perform (\rsec{IO_iohints_type}).
      If \chpl{hints} is \chpl{IOHINT_NONE}, the hints provided when opening
      the file will be used.

\item \chpl{style} defines this channel's I/O style (\rsec{IO_io_style}).
      If the corresponding actual argument is omitted,
      the I/O style will be set to the style specified when the file
      was opened (\rsec{IO_open_file}).

\end{itemize}
\end{protobody}

\index{input/output!reader@\chpl{reader}}
\begin{protohead}
proc file.reader(out error:syserr, param kind=iokind.dynamic, param locking=true,
                 start:int(64) = 0, end:int(64) = max(int(64)), hints:iohints = IOHINT_NONE,
                 style:iostyle = this._style): channel(false, kind, locking);
proc file.reader(param kind=iokind.dynamic, param locking=true,
                 start:int(64) = 0, end:int(64) = max(int(64)), hints:iohints = IOHINT_NONE,
                 style:iostyle = this._style): channel(false, kind, locking);
\end{protohead}
\begin{protobody}
Creates a channel for reading from the file.

The arguments are the same as for \chpl{file.writer}, except for the following:

\begin{itemize}

\item \chpl{start} and \chpl{end} define the region of the file that
      the channel will read from. The default values for these arguments enable
      the channel to access the entire file.
      A channel will never read beyond its maximum end position.

\end{itemize}
\end{protobody}


\subsection{Synchronization of Channel Data}
\label{IO_channel_synchronization}

The following synchronization operation is available.

\index{input/output!flush@\chpl{flush}}
\begin{protohead}
proc channel.flush(out error:syserr);
proc channel.flush();
\end{protohead}
\begin{protobody}
Makes all writes to the channel, if any, available to concurrent viewers
of its associated file, such as other channels or other applications
accessing this file concurrently.
Unlike \chpl{file.fsync}, this does not commit the written data
to the file's device.
\end{protobody}


\subsection{Functions for Closing Channels}
\label{IO_closing_channels}

A channel must be closed in order to free the resources allocated for it,
to ensure that data written to it is visible to other channels,
or to allow the associated file to be closed.

\index{input/output!close (channel)@\chpl{close} (channel)}
\begin{protohead}
proc channel.close(out error:syserr);
proc channel.close();
\end{protohead}
\begin{protobody}
Close a channel after performing the \chpl{flush} operation
(\rsec{IO_channel_synchronization}).
The \chpl{error} argument is explained in \rsec{IO_error_handling}.
\end{protobody}

%TODO: can one be closed multiple times?
It is an error to perform any I/O operations on a channel
that has been closed.
It is an error to close a file when it has channels that
have not been closed.

\begin{future}
In the future we plan to implement reference counting for files
and channels. Each file and channel will be closed automatically
when no references remain to it. For example, if only a local
variable refers to a channel, the channel will be closed
when that variable goes out of scope.
% any more here?
The ability for the program to close a file or a channel
explicitly will remain available.
\end{future}


\subsection{I/O Style}
\label{IO_io_style}
\index{input/output!I/O style}

I/O style regulates how Chapel's basic types are read or written.
For example, the I/O style defines whether I/O is binary or text
and what representation is used for each basic type.

Section~\rsec{IO_iostyle_type} defines the \chpl{iostyle} type
used to represent I/O styles and provides details on formatting
and other representation choices.

\index{input/output!I/O style!applicable}
Each write/read operation has an \emph{applicable} I/O style.
It is defined by the \chpl{iostyle} argument if one is passed explicitly
when invoking the corresponding write/read function.
Otherwise it is the channel's I/O style, which is established
upon channel creation (\rsec{IO_channel_creation}).


\subsection{The {\em stdin}, {\em stdout}, and {\em stderr} Channels}
\label{IO_std_channels}
\index{input/output!stdin, stdout, stderr@\chpl{stdin}, \chpl{stdout}, \chpl{stderr}}

Chapel provides the predefined channels \chpl{stdin}, \chpl{stdout},
and \chpl{stderr} to access the corresponding operating system streams
standard input, standard output, and standard error.

\chpl{stdin} supports reading;
\chpl{stdout} and \chpl{stderr} support writing.
All three channels are safe to use concurrently.
Their types' \chpl{kind} argument is \chpl{dynamic}.


\section{Supporting Data Types}
\label{IO_data_types}

File and channel operations make use of the types presented in this section.


\subsection{The {\em iokind} Type}
\label{IO_iokind_type}
\index{input/output!iokind@\chpl{iokind}}
\index{input/output!types!iokind@\chpl{iokind}}

The \chpl{iokind} type is an enum. When used
as arguments to the \chpl{channel} type,
its constants have the following meaning:

\begin{itemize}

\item \chpl{big} binary I/O with big-endian byte order
      is performed when writing/reading basic types from the channel.

\item \chpl{little} similar to \chpl{big}, but with little-endian byte order.

\item \chpl{native} similar to \chpl{big}, but with the byte order
      that is native to the target platform.

\item \chpl{dynamic} means that the applicable I/O style
      has full effect.

\end{itemize}

In the case of \chpl{big}, \chpl{little}, and \chpl{native},
the applicable I/O style is consulted when writing/reading
strings, but not for other basic types.


\subsection{The {\em iomode} Type}
\label{IO_iomode_type}
\index{input/output!iomode@\chpl{iomode}}
\index{input/output!types!iomode@\chpl{iomode}}

The \chpl{iomode} type is an enum.
When used as arguments when opening files,
its constants have the following meaning:

\begin{itemize}

\item \chpl{r} - open an existing file for reading.

\item \chpl{rw} - open an existing file for reading and writing.

\item \chpl{cw} - create a new file for writing.
      If the file already exists, its contents are removed
      when the file is opened in this mode.

\item \chpl{cwr} is like \chpl{cw},
      but reading from the file is also allowed.

\end{itemize}


\subsection{The {\em iohints} Type}
\label{IO_iohints_type}
\index{input/output!iohints@\chpl{iohints}}
\index{input/output!types!iohints@\chpl{iohints}}

A value of the \chpl{iohints} type defines a set of hints
about the I/O that the file or channel will perform.
These hints may be used by the implementation
to select optimized versions of the I/O operations.

The \chpl{iohints} type is implementation-defined.
The following \chpl{iohints} constants are provided:

\begin{itemize}

\item \chpl{IOHINT_NONE} defines an empty set, which provides no hints.
%include this?  This means normal operation, nothing special to hint about.

\item \chpl{IOHINT_CACHED} suggests that the file data is or should be
      cached in memory, possibly all at once.

\item \chpl{IOHINT_RANDOM} suggests to expect random access.

\item \chpl{IOHINT_SEQUENTIAL} suggests to expect sequential access.

\item \chpl{IOHINT_PARALLEL} suggests to expect many channels
      working with this file in parallel.

\end{itemize}

\begin{future}
Other hints are likely to be added in the future.
\end{future}

The following binary operators are defined on \chpl{iohints}:

\begin{itemize}

\item \chpl{|} set union,

\item \chpl{&} set intersection,

\item \chpl{==} set equality,

\item \chpl{!=} set inequality.

\end{itemize}

When an \chpl{iohints} formal has blank intent, the
actual is copied to the formal upon a function call and
the formal cannot be assigned within the function.

%\index{input/output!iohints@\chpl{iohints}!default value}
The default value of the \chpl{iohints} type is undefined.


\subsection{The {\em iostyle} Type}
\label{IO_iostyle_type}
\index{input/output!iostyle@\chpl{iostyle}}
\index{input/output!types!iostyle@\chpl{iostyle}}
\index{input/output!I/O style!iostyle@\chpl{iostyle}}

The \chpl{iostyle} type represents I/O styles \rsec{IO_io_style},
defining how Chapel's basic types should be read or written.

The \chpl{iostyle} type is an implementation-defined record.

\begin{craychapel}
The \chpl{iostyle} type is presented in
\chpl{CHPL_HOME/doc/technotes/README.io}.
\end{craychapel}

\begin{future}
In the future we may specify additional requirements, such as
the required \chpl{iostyle} fields and methods.
We also plan to provide a method to retrieve a channel's I/O style.
\end{future}


\subsection{The {\em syserr} Type}
\label{IO_syserr_type}
\index{input/output!syserr@\chpl{syserr}}

The \chpl{syserr} type is used to represent success or an error
condition of most Chapel I/O functions, which is accessible as
described in \rsec{IO_error_handling}.

Success is represented by the predefined constant \chpl{ENOERR}.
The other values represent errors.
The constant \chpl{EEOF} represents the unexpected-end-of-file error.

A string describing the error can be obtained using the following function:

\begin{protohead}
proc errorToString(error:syserr):string;
\end{protohead}
\begin{protobody}
Returns a string describing the success or error condition in \chpl{error}.
\end{protobody}

\begin{craychapel}
Other \chpl{syserr} constants are described in
\\ %formatting
\chpl{CHPL_HOME/doc/technotes/README.io}.
\end{craychapel}

The \chpl{syserr} type is implementation-defined.

When a \chpl{syserr} formal has blank intent, the
actual is copied to the formal upon a function call and
the formal cannot be assigned within the function.

%\index{input/output!syserr@\chpl{syserr}!default value}
The default value of the \chpl{syserr} type is undefined.


\section{Error Handling}
\label{IO_error_handling}
\index{input/output!error handling}
\index{input/output!syserr@\chpl{syserr}}

Most I/O functions have two versions: with the \chpl{out error:syserr}
argument and without. The version that takes the \chpl{error} argument
returns the success or error code (\rsec{IO_syserr_type}) in that argument.
The version that does not accept \chpl{error} halts with an error message
if an error condition is encountered.

\begin{future}
If exceptions are added to Chapel, the no-\chpl{error} versions
could be changed to throw exceptions instead of halting.
\end{future}

In most cases I/O errors are reported by the function during whose
call those errors occurred. However, in some cases errors are
reported only upon \chpl{file.close} \rsec{IO_closing_files}
or \chpl{file.fsync} \rsec{IO_file_synchronization}.
Therefore, one of these two functions must be invoked to ensure
that the data has arrived on disk successfully or an error is reported.


% TODO: the following is retained from the previous version of
% the I/O chapter with only minor changes. May need more rewriting.

\section{Writing and Reading}
\label{IO_writing_reading}


\subsection{The {\em write}, {\em writeln}, {\em read}, and {\em readln} 
Functions}
\index{writeln@\chpl{writeln}}
\index{write@\chpl{write}}
\index{read@\chpl{read}}
\index{readln@\chpl{readln}}

The predefined function \chpl{write} takes an arbitrary number of
arguments and prints each out in turn to \chpl{stdout}.  The predefined
function \chpl{writeln} is identical to \chpl{write} except that it
outputs an additional {\em end-of-line} character after writing out
the argument expressions.  Both of these functions will generate their
output atomically with respect to other calls to these functions from
other tasks.

The predefined function \chpl{read} takes an arbitrary number of
variable expressions and reads into each in turn from \chpl{stdin}.
For text I/O, any whitespace is skipped over and is used only to separate one
argument from the next.  The predefined function \chpl{readln} is
identical except that upon reading all of its arguments it scans ahead
in the input stream until just after the next {\em end-of-line}
character.

The \chpl{read} and \chpl{readln} functions are also defined to take
an arbitrary number of types as arguments.  In this case, the
functions read an expression of each argument type.  In the event that
a single type is specified, the return value is the value that was
read; if multiple types are specified, a tuple of the values is
returned.

These functions are provided for convenience.
\chpl{write} and \chpl{writeln} invoke the correspondingly-named methods
on the \chpl{stdout} channel (\rsec{IO_channel_write}).
\chpl{read} and \chpl{readln} invoke the correspondingly-named methods
on the \chpl{stdin} channel (\rsec{IO_channel_read}).

\begin{chapelexample}{writeln.chpl}
The \chpl{writeln} function allows for a simple implementation
of the {\em Hello-World} program:
\begin{chapel}
writeln("Hello, World!");
\end{chapel}
\begin{chapelprintoutput}
Hello, World!
\end{chapelprintoutput}
\end{chapelexample}

\begin{example}
The following code shows three ways to read values into a pair of
variables \chpl{x} and \chpl{y}:
\begin{chapel}
var x: int;
var y: real;

/* reading into variable expressions */
read(x, y);

/* reading via a single type argument */
x = read(int);
y = read(real);

/* reading via multiple type arguments */
(x, y) = read(int, real);
\end{chapel}
\end{example}

\label{User_Defined_writeThis_Methods}
\index{input/output!writeThis@\chpl{writeThis}}

\subsection{The {\em write} and {\em writeln} Methods on Channels}
\label{IO_channel_write}
\index{write!on channels}
\index{write (channel)@\chpl{write} (channel)}
\index{writeln (channel)@\chpl{writeln} (channel)}
\index{input/output!write (channel)@\chpl{write} (channel)}
\index{input/output!writeln (channel)@\chpl{writeln} (channel)}

The \chpl{channel} type supports methods \chpl{write} and \chpl{writeln}
for output.  These methods are defined to take an arbitrary number of
arguments.  Each argument is written in turn by calling
the \chpl{writeThis} method on that argument.
Default \chpl{writeThis} methods are bound to any type that the user
does not explicitly create one for.

Like most other methods on channels,
these operations are guarded against concurrent execution on the same channel
when the channel type's \chpl{locking} parameter is \chpl{true}.


\subsection{The {\em read} and {\em readln} Methods on Channels}
\label{IO_channel_read}
\index{input/output!read@\chpl{read}}
\index{input/output!readln@\chpl{readln}}
\index{read!on channels}

The \chpl{channel} type supports \chpl{read} and \chpl{readln} methods.
The \chpl{read} method takes an arbitrary number of arguments, reading
in each argument from channel.  The \chpl{readln} method also
takes an arbitrary number of arguments, reading in each argument
from a single line or multiple lines in the channel and 
advancing the channel pointer to the next line after the last argument 
is read.

The \chpl{channel} type also supports overloaded methods \chpl{read}
and \chpl{readln} that take an arbitrary number of types as arguments.
These methods read values of the specified types from the channel and
return them in a tuple.  If only one type is read, the value is not
returned in a tuple, but is returned directly.

\begin{example}
The following line of code reads a value of type \chpl{int} from
\chpl{stdin} and uses it to initialize variable \chpl{x} (causing
\chpl{x} to have an inferred type of \chpl{int}):
\begin{chapel}
var x = stdin.read(int);
\end{chapel}
\end{example}


\subsection{The {\em write} and {\em writeln} Methods on Strings}
\label{stringwrite}
\index{write!on strings}
\index{write (string)@\chpl{write} (string)}
\index{writeln (string)@\chpl{writeln} (string)}
\index{input/output!write (string)@\chpl{write} (string)}
\index{input/output!writeln (string)@\chpl{writeln} (string)}

The \chpl{write} and \chpl{writeln} methods can also be called on
strings to write the output to a string instead of a channel.


%?? \subsection{The {\em read} and {\em readln} method on strings}


\subsection{The {\em readThis}, {\em writeThis}, and {\em readWriteThis} Methods}

When programming the input and output method for a custom data type, it is often useful to define both the read and write routines at the same time. That is possible to do in a Chapel program by defining a \chpl{readWriteThis} method, which is a generic method expecting a single argument: either a Reader or a Writer.

In cases when the reading routine and the writing routine are more naturally separate, or in which only one should be defined, a Chapel program can define \chpl{readThis} (taking in a single argument of type Reader) and/or \chpl{writeThis} (taking in a single argument of type Writer). 

If a none of these routines are provided, a default version of \chpl{readThis} and \chpl{writeThis} will be generated by the compiler. If \chpl{readWriteThis} is defined, the compiler will generate \chpl{readThis} or \chpl{writeThis} - if they do not already exist - which call \chpl{readWriteThis}.

Objects of type \chpl{Reader} or \chpl{Writer} both support the following methods:
\begin{chapel}
  /* Return true if we are a Writer (vs a Reader) */
  proc writing: bool;
  /* Return true if we are doing binary I/O */
  proc binary: bool;
  /* Return the current error */
  proc error():syserr;
  /* Set the current error */
  proc setError(e:syserr);
  /* Clear the current error */
  proc clearError();
  /* Read or write a value according to its readThis or writeThis method */
  proc readwrite(ref x);
\end{chapel}

Objects of type \chpl{Reader} also supports the following methods:
\begin{chapel}
  /* as with channel.read */
  proc read(ref args ...):bool;
  /* as with channel.readln */
  proc readln(ref args ...):bool;
  /* as with channel.readln */
  proc readln():bool;
\end{chapel}
Objects of type \chpl{Writer} also support the following methods:
\begin{chapel}
  /* as with channel.write */
  proc write(args ...);
  /* as with channel.writeln */
  proc writeln(args ...);
  /* as with channel.writeln */
  proc writeln();
\end{chapel}

Note that objects of type \chpl{Reader} or \chpl{Writer} may represent a locked channel; as a result, using parallelism constructs to call methods on Reader or Writer may result in undefined behavior.

Because it is often more convenient to use an operator for I/O, instead of writing \chpl{f.readwrite(x); f.readwrite(y);}, one can write \chpl{f <~> x <~> y;}. Note that the types \chpl{ioLiteral} and \chpl{ioNewline} may be useful when using the \chpl{<~>}. \chpl{ioLiteral} represents some string that must be read or written as-is (e.g. \chpl{","} when working with a tuple), and \chpl{ioNewline} will emit a newline when writing but skip to and consume a newline when reading.

\begin{chapelexample}{UserReadWrite.chpl}
This example defines a readWriteThis method and demonstrates how \chpl{<~>} will call the read or write routine, depending on the situation.
\begin{chapel}
class IntPair {
  var x: int;
  var y: int;
  proc readWriteThis(f) {
    f <~> x <~> new ioLiteral(",") <~> y <~> new ioNewline();
  }
}
var ip = new IntPair(17,2);
write(ip);
\end{chapel}
\begin{chapelprintoutput}
17,2
\end{chapelprintoutput}
\end{chapelexample}

\begin{chapelexample}{UserWrite.chpl}
This example defines a only a writeThis method - so that there will be a function resolution error if the class NoRead is read.

\begin{chapel}
class NoRead {
  var x: int;
  var y: int;
  proc writeThis(f:Writer) {
    f.writeln("hello");
  }
  // Note that no readThis function will be generated.
}
var nr = new NoRead();
write(nr);
// Note that read(nr) will generate a compiler error.
\end{chapel}
\begin{chapelprintoutput}
hello
\end{chapelprintoutput}
\end{chapelexample}


\subsection{Generalized {\em write} and {\em writeln}}
\label{writer}
\index{Writer@\chpl{Writer}}

The \chpl{Writer} class contains no arguments and serves as a base
class to allow user-defined classes to be written to.  If a class is
defined to be a subclass of Writer, it must override
the \chpl{writeIt} method that takes a \chpl{string} as an argument.

\begin{chapelexample}{UserWriter.chpl}
The following code defines a subclass of \chpl{Writer} that overrides
the \chpl{writeIt} method to allow it to be written to.  It also
overrides the \chpl{writeThis} method to override the default way that
it is written.
\begin{chapel}
class C: Writer {
  var data: string;
  proc writePrimitive(x) {
    var s = x:string;
    data += s.substring(1);
  }
  proc writeThis(x: Writer) {
    x.write(data);
  }
}

var c = new C();
c.write(41, 32, 23, 14);
writeln(c);
\end{chapel}
The \chpl{C} class filters the arguments sent to it, printing out only
the first letter.  The output to the above is thus:
% TODO: when 'chapelprintoutput' is extended with some wording,
% move things around so that the printed text is well-composed.
\begin{chapelprintoutput}
4321
\end{chapelprintoutput}
\end{chapelexample}


%TODO: \subsection{Generalized {\em read} and {\em readln}}


%TODO: define the error behavior of read/write/readThis/writeThis/Reader/Writer


\subsection{Default {\em write} and {\em read} Methods}
\index{write@\chpl{write}!default method}
\index{read@\chpl{read}!default method}

% TODO: these should be moved to the chapters for the corresponding
% data types

% TODO: these are methods on channels, taking arguments of all types.
% Need to make that clear in the text.

Default \chpl{write} methods are created for all types for which a user-defined
\chpl{write} method is not provided.  They have the following semantics:
\begin{itemize}
\item
{\bf arrays} Outputs the elements of the array in row-major order
where rows are separated by line-feeds and blank lines are used to
separate other dimensions.
\item
{\bf domains} Outputs the dimensions of the domain enclosed
by \chpl{[} and \chpl{]}.
\item
{\bf ranges} Outputs the lower bound of the range followed
by \chpl{..} followed by the upper bound of the range.  If the stride
of the range is not one, the output is additionally followed by the
word \chpl{by} followed by the stride of the range.
\item
{\bf tuples} Outputs the components of the tuple in order delimited
by \chpl{(} and \chpl{)}, and separated by commas.
\item
{\bf classes} Outputs the values within the fields of the class
prefixed by the name of the field and the character \chpl{=}.  Each
field is separated by a comma.  The output is delimited by \chpl{\{}
and \chpl{\}}.
\item
{\bf records} Outputs the values within the fields of the class
prefixed by the name of the field and the character \chpl{=}.  Each
field is separated by a comma.  The output is delimited by \chpl{(}
and \chpl{)}.
\end{itemize}

Default \chpl{read} methods are created for all types for which a user-defined
\chpl{read} method is not provided.  The default \chpl{read} methods are
defined to read in the output of the default \chpl{write} method.
