\sekshun{Iterators}
\label{Iterators}
\index{iterators}

An iterator is a function that conceptually returns multiple values
rather than simply a single value.

\begin{openissue}
The parallel iterator story is under development.  It is expected that
the specification will be expanded regarding parallel iterators soon.
\end{openissue}

\subsection{Iterator Functions}
\label{Iterator_Functions}

The syntax of an iterator declaration is identical to that of a
function declaration.  A function is an iterator if it includes yield
statements.  When a yield is encountered, the value is returned, but
the iterator is not finished evaluating when called within a loop.  It
will continue from the point after the yield and can yield or return
more values.  When a return is encountered, the value is returned and
the iterator finishes.  An iterator also completes after the last
statement in the iterator function is executed.

\subsection{The Yield Statement}
\label{The_Yield_Statement}
\index{yield@\chpl{yield}}

The yield statements can only appear in iterators.  The syntax of the
yield statement is given by
\begin{syntax}
yield-statement:
  `yield' expression ;
\end{syntax}

\subsection{Iterator Calls}
\label{Iterator_Calls}

Iterator functions can be called within for or forall loops, in which
case they are executed in an interleaved manner with the body of the
loop, can be captured in new variable declarations or arrays, in which
case they evaluate to an array of values, or can be passed to a
generic function argument.

\subsubsection{Iterators in For and Forall Loops}
\label{Iterators_in_For_and_Forall_Loops}

When an iterator is accessed via a for or forall loop, the iterator is
evaluated alongside the loop body in an interleaved manner.  For each
iteration, the iterator yields a value and the body is executed.

\subsubsection{Iterators as Arrays}
\label{Iterators_as_Arrays}
\index{iterators!and arrays}

If an iterator function is captured into a new variable declaration or
assigned to an array, the iterator is iterated over in total and the
expression evaluates to a one-dimensional arithmetic array that
contains the values returned by the iterator on each iteration.
\begin{example}
Given an iterator
\begin{chapel}
def squares(n: int): int {
  for i in 1..n do
    yield i * i;
}
\end{chapel}
the expression \chpl{squares(5)} evaluates to the array \chpl{1, 4, 9, 16, 25}.
\end{example}

\subsubsection{Iterators and Generics}
\label{Iterators_and_Generics}
\index{iterators!and generics}

If an iterator call expression is passed to a function argument that
is generic, the iterator is passed without being evaluated and is
treated as a closure within the generic function.

\subsection{Parallel Iterators}
\label{Parallel_Iterators}

Iterators used in explicit forall-statements or -expressions must be
parallel iterators.  Reductions, scans and promotion over serial
iterators will be serialized.

The definition of parallel iterators is forthcoming.  Parallel
iterators are defined over standard constructs in Chapel such as
ranges, domains, and arrays (including Block- and Cyclic-distributed
domains and arrays), thereby allowing these constructs to be used with
forall-statements and -expressions.
