\sekshun{Language Overview}
\label{Language_Overview}

Chapel is a new programming language under development at Cray Inc. as
part of the DARPA High Productivity Computing Systems (HPCS) program
to improve the productivity of parallel programmers.

This section provides a brief overview of the Chapel language by
discussing first the guiding principles behind the design of the
language and second how to get started with Chapel.

\section{Guiding Principles}

The following four principles guided the design of Chapel:
\begin{enumerate}
\item General parallel programming
\item Locality-aware programming
\item Object-oriented programming
\item Generic programming
\end{enumerate}
The first two principles were motivated by a desire to support
general, performance-oriented parallel programming through high-level
abstractions.  The second two principles were motivated by a desire to
narrow the gulf between high-performance parallel programming
languages and mainstream programming and scripting languages.

\subsection{General Parallel Programming}

First and foremost, Chapel is designed to support general parallel
programming through the use of high-level language abstractions.
Chapel supports a \emph{global-view programming model} that raises the
level of abstraction in expressing both data and control flow as
compared to parallel programming models currently in use.
A global-view programming model is best defined in terms
of \emph{global-view data structures} and a \emph{global view of
control}.

\emph{Global-view data structures} are arrays and other data
aggregates whose sizes and indices are expressed globally even though
their implementations may distribute them across the \emph{locales} of
a parallel system.  A locale is an abstraction of a unit of uniform
memory access on a target architecture.  That is, within a locale, all
threads exhibit similar access times to any specific memory address.
For example, a locale in a commodity cluster could be defined to be a
single core of a processor, a multicore processor or an SMP node of
multiple processors.

Such a global view of data contrasts with most parallel languages
which tend to require users to partition distributed data aggregates
into per-processor chunks either manually or using language
abstractions.  As a simple example, consider creating a 0-based vector
with $n$ elements distributed between $p$ locales.  A language like
Chapel that supports global-view data structures allows the user to
declare the array to contain $n$ elements and to refer to the array
using the indices $0 \ldots n-1$.  In contrast, most traditional
approaches require the user to declare the array as $p$ chunks of
$n/p$ elements each and to specify and manage inter-processor
communication and synchronization explicitly (and the details can be
messy if $p$ does not divide $n$ evenly).  Moreover, the chunks are
typically accessed using local indices on each processor
(\eg,~$0..n/p$), requiring the user to explicitly translate between
logical indices and those used by the implementation.

A \emph{global view of control} means that a user's program commences
execution with a single logical thread of control and then introduces
additional parallelism through the use of certain language concepts.
All parallelism in Chapel is implemented via multithreading, though
these threads are created via high-level language concepts and managed
by the compiler and runtime rather than through explicit
fork/join-style programming.  An impact of this approach is that
Chapel can express parallelism that is more general than the Single
Program, Multiple Data~(SPMD) model that today's most common parallel
programming approaches use.  Chapel's general support for parallelism does not
preclude users from coding in an SPMD style if they wish.

Supporting general parallel programming also means targeting a broad
range of parallel architectures.  Chapel is designed to target a wide
spectrum of HPC hardware including clusters of commodity processors
and SMPs; vector, multithreading, and multicore processors; custom
vendor architectures; distributed-memory, shared-memory, and 
shared address-space architectures; and networks of any topology.  Our
portability goal is to have any legal Chapel program run correctly on
all of these architectures, and for Chapel programs that express
parallelism in an architecturally-neutral way to perform reasonably on
all of them.  Naturally, Chapel programmers can tune their code to
more closely match a particular machine's characteristics.

\subsection{Locality-Aware Programming}

A second principle in Chapel is to allow the user to optionally and
incrementally specify where data and computation should be placed on
the physical machine.  Such control over program locality is essential
to achieve scalable performance on distributed-memory architectures.  Such control
contrasts with shared-memory programming models which present the user
with a simple flat memory model.  It also contrasts with SPMD-based
programming models in which such details are explicitly specified by
the programmer on a process-by-process basis via the multiple
cooperating program instances.

\subsection{Object-Oriented Programming}

A third principle in Chapel is support for object-oriented
programming.  Object-oriented programming has been instrumental in
raising productivity in the mainstream programming community due to
its encapsulation of related data and functions within a single software
component, its support for specialization and reuse, and its use as a
clean mechanism for defining and implementing interfaces.  Chapel
supports objects in order to make these benefits available in a
parallel language setting, and to provide a familiar coding paradigm for
members of the mainstream programming community.  Chapel supports
traditional reference-based classes as well as value classes that are
assigned and passed by value.

Chapel does not require the programmer to use an object-oriented style
in their code, so that traditional Fortran and C programmers in the
HPC community need not adopt a new programming paradigm to
use Chapel effectively.  Many of Chapel's standard library
capabilities are implemented using objects, so such programmers may
need to utilize a method-invocation style of syntax to use these
capabilities.  However, using such libraries does not necessitate
broader adoption of object-oriented methodologies.

\subsection{Generic Programming}

Chapel's fourth principle is support for generic programming and
polymorphism.  These features allow code to be written in a style that
is generic across types, making it applicable to variables of multiple
types, sizes, and precisions.  The goal of these features is to
support exploratory programming as in popular interpreted and
scripting languages, and to support code reuse by allowing algorithms
to be expressed without explicitly replicating them for each possible
type.  This flexibility at the source level is implemented by having
the compiler create versions of the code for each required type
signature rather than by relying on dynamic typing which would result
in unacceptable runtime overheads for the HPC community.

\section{Getting Started}

A Chapel version of the standard ``hello, world'' computation is as
follows:
\vspace{0.5pc}
\begin{chapel}
writeln("hello, world");
\end{chapel}
This complete Chapel program contains a single line of code that makes
a call to the standard \chpl{writeln} function.

\index{modules}
\index{main@\chpl{main}}

In general, Chapel programs define code using one or more named
\emph{modules}, each of which supports top-level initialization code
that is invoked the first time the module is used.  Programs also
define a single entry point via a function named \chpl{main}.  To
facilitate exploratory programming, Chapel allows programmers to
define modules using files rather than an explicit module declaration
and to omit the program entry point when the program only has a single
user module.

Chapel code is stored in files with the extension \chpl{.chpl}.
Assuming the ``hello, world'' program is stored in a file
called \chpl{hello.chpl}, it would define a single user
module, \chpl{hello}, whose name is taken from the filename.  Since
the file defines a module, the top-level code in the file defines the
module's initialization code.  And since the program is composed of
the single \chpl{hello} module, the \chpl{main} function is omitted.
Thus, when the program is executed, the single \chpl{hello} module
will be initialized by executing its top-level code thus invoking the
call to the \chpl{writeln} function.  Modules are described in more
detail in~\rsec{Modules}.

To compile and run the ``hello world'' program, execute the following
commands at the system prompt:
\begin{commandline} 
> chpl hello.chpl
> ./a.out
\end{commandline}
The following output will be printed to the console:
\begin{commandline}
hello, world
\end{commandline}
