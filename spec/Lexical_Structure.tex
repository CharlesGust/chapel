\sekshun{Lexical Structure}
\label{Lexical_Structure}
\index{lexical structure}

This section describes the lexical components of Chapel programs.
The purpose of lexical analysis is
to separate the raw input stream into a sequence of tokens suitable
for input to the parser.  

\section{Comments}
\label{Comments}
\index{comments}
\index{lexical structure!comments}

Two forms of comments are supported.  All text following the
consecutive characters {\tt //} and before the end of the line is in a
comment.  All text following the consecutive characters {\tt /*} and
before the consecutive characters {\tt */} is in a comment.
A comment delimited by {\tt /*} and {\tt */} can be nested in
another comment delimited by {\tt /*} and {\tt */}

Comments, including the characters that delimit them, do not affect
the behavior of the program (except in delimiting tokens).  If the
delimiters that start the comments appear within a string literal,
they do not start a comment but rather are part of the string literal.

\begin{example}
The following program makes use of both forms of comment:
\begin{chapel}
/*
 *  main function
 */
proc main() {
  writeln("hello, world"); // output greeting with new line
}
\end{chapel}
\end{example}

\section{White Space}
\label{White_Space}
\index{white space}
\index{lexical structure!white space}

White-space characters are spaces, tabs, line feeds, and carriage
returns.  Along with comments, they delimit tokens, but are otherwise
ignored.

\section{Case Sensitivity}
\label{Case_Sensitivity}
\index{case sensitivity}
\index{lexical structure!case sensitivity}

Chapel is a case sensitive language.

\begin{example}
The following identifiers are considered
distinct: \chpl{chapel}, \chpl{Chapel}, and \chpl{CHAPEL}.
\end{example}

\section{Tokens}
\label{Tokens}
\index{lexical structure!tokens}

Tokens include identifiers, keywords, literals, operators, and
punctuation.

\subsection{Identifiers}
\label{Identifiers}
\index{identifiers}
\index{lexical structure!identifiers}

An identifier in Chapel is a sequence of characters that starts with a
lowercase or uppercase letter or an underscore and is optionally
followed by a sequence of lowercase or uppercase letters, digits,
underscores, and dollar-signs.  Identifiers are designated by the
following syntax:
\begin{syntax}
identifier:
  letter-or-underscore legal-identifier-chars[OPT]

legal-identifier-chars:
  legal-identifier-char legal-identifier-chars[OPT]

legal-identifier-char:
  letter-or-underscore
  digit
  `(*\texttt{\$}*)'

letter-or-underscore:
  letter
  `_'

letter: one of
  `A' `B' `C' `D' `E' `F' `G' `H' `I' `J' `K' `L' `M' `N' `O' `P' `Q' `R' `S' `T' `U' `V' `W' `X' `Y' `Z'
  `a' `b' `c' `d' `e' `f' `g' `h' `i' `j' `k' `l' `m' `n' `o' `p' `q' `r' `s' `t' `u' `v' `w' `x' `y' `z'

digit: one of
  `0' `1' `2' `3' `4' `5' `6' `7' `8' `9'
\end{syntax}

\begin{rationale}
Why include ``\$'' in the language?  The inclusion of the \$ character
is meant to assist programmers using sync and single variables by
supporting a convention (a \$ at the end of such variables) in order
to help write properly synchronized code.  It is felt that marking
such variables is useful since using such variables could result in
deadlocks.
\end{rationale}

\begin{example}
The following are legal
identifiers: \chpl{Cray1}, \chpl{syncvar$\mbox{\texttt{\$}}$},
\chpl{legalIdentifier}, and \chpl{legal_identifier}.
\end{example}

\subsection{Keywords}
\label{Keywords}
\index{keywords}
\index{lexical structure!keywords}

The following identifiers are reserved as keywords:

\begin{tabular}{llllll}
&
\begin{chapel}
_
align
atomic
begin
break
by
class
cobegin
coforall
config
const
continue
delete
\end{chapel} & \begin{chapel}
dmapped
do
domain
else
enum
export
extern
for
forall
if
in
index
\end{chapel} & \begin{chapel}
inline
inout
iter
label
let
local
module
new
nil
noinit
on
otherwise
\end{chapel} & \begin{chapel}
out
param
proc
record
reduce
ref
return
scan
select
serial
single
sparse
\end{chapel} & \begin{chapel}
subdomain
sync
then
type
union
use
var
when
where
while
yield
zip
\end{chapel} \\
\begin{invisible}
otherwise
\end{invisible} & \begin{invisible}
otherwise
\end{invisible} & \begin{invisible}
otherwise
\end{invisible} & \begin{invisible}
otherwise
\end{invisible} & \begin{invisible}
otherwise
\end{invisible} & \begin{invisible}
otherwise
\end{invisible}
\end{tabular}

The following identifiers are keywords reserved for future use:

\begin{tabular}{l}
\begin{chapel}
lambda
\end{chapel} \\
\begin{invisible}
otherwise
\end{invisible}
\end{tabular}

\subsection{Literals}
\label{Literals}
\label{Primitive_Type_Literals}
\index{lexical structure!literals}

\index{literals!primitive type}
Bool literals are designated by the following syntax:
\begin{syntax}
bool-literal: one of
  `true' $ $ $ $ `false'
\end{syntax}

Signed and unsigned integer literals are designated by the following
syntax:
\begin{syntax}
integer-literal:
  digits
  `0x' hexadecimal-digits
  `0X' hexadecimal-digits
  `0b' binary-digits
  `0B' binary-digits

digits:
  digit
  digit digits

hexadecimal-digits:
  hexadecimal-digit
  hexadecimal-digit hexadecimal-digits

hexadecimal-digit: one of
  `0' `1' `2' `3' `4' `5' `6' `7' `8' `9' `A' `B' `C' `D' `E' `F' `a' `b' `c' `d' `e' `f'

binary-digits:
  binary-digit
  binary-digit binary-digits

binary-digit: one of
  `0' `1'
\end{syntax}

All integer literals have type \chpl{int}.

\begin{rationale}
Why are there no suffixes on integral literals?  Suffixes, like those
in C, are not necessary.  Explicit
conversions can then be used to change the type of the literal to
another integer size.
\end{rationale}

Real literals are designated by the following syntax:
\begin{syntax}
real-literal:
  digits[OPT] . digits exponent-part[OPT]
  digits .[OPT] exponent-part

exponent-part:
  `e' sign[OPT] digits
  `E' sign[OPT] digits

sign: one of
  + $ $ $ $ -
\end{syntax}
\begin{rationale}
Why can't a real literal end with '.'?  There is a lexical ambiguity
between real literals ending in '.' and the range operator '..' that
makes it difficult to parse.  For example, we want to
parse \chpl{1..10} as a range from 1 to 10 without concern
that \chpl{1.} is a real literal.
\end{rationale}
The type of a real literal is \chpl{real}.  Explicit conversions are
necessary to change the size of the literal.

Imaginary literals are designated by the following syntax:
\begin{syntax}
imaginary-literal:
  real-literal `i'
  integer-literal `i'
\end{syntax}
The type of an imaginary literal is \chpl{imag}.  Explicit conversions
are necessary to change the size of the literal.

There are no complex literals.  Rather, a complex value can be
specified by adding or subtracting a real literal with an imaginary
literal.  Alternatively, a 2-tuple of integral or real expressions can
be cast to a complex such that the first component becomes the real
part and the second component becomes the imaginary part.
\begin{example}
The following expressions are identical: \chpl{1.0 + 2.0i}
and \chpl{(1.0, 2.0):complex}.
\end{example}

String literals are designated by the following syntax:
\begin{syntax}
string-literal:
  " double-quote-delimited-characters[OPT] "
  ' single-quote-delimited-characters[OPT] '

double-quote-delimited-characters:
  string-character double-quote-delimited-characters[OPT]
  ' double-quote-delimited-characters[OPT]

single-quote-delimited-characters:
  string-character single-quote-delimited-characters[OPT]
  " single-quote-delimited-characters[OPT]

string-character:
  `any character except the double quote, single quote, or new line'
  simple-escape-character
  hexadecimal-escape-character

simple-escape-character: one of
  `$\backslash\mbox{\bf '}\hspace{5pt}$' `$\backslash$"$\hspace{5pt}$' `$\backslash$?$\hspace{5pt}$' `$\backslash$a$\hspace{5pt}$' `$\backslash$b$\hspace{5pt}$' `$\backslash$f$\hspace{5pt}$' `$\backslash$n$\hspace{5pt}$' `$\backslash$r$\hspace{5pt}$' `$\backslash$t$\hspace{5pt}$' `$\backslash$v$\hspace{5pt}$'

hexadecimal-escape-character:
  `$\backslash$x' hexadecimal-digits
\end{syntax}

\subsection{Operators and Punctuation}
\label{Operators_and_Punctuation}
\index{lexical structure!operator}
\index{operators!lexical structure}
\index{lexical structure!punctuation}

The following operators and punctuation are defined in the syntax of
the language:
\begin{center}
\begin{tabular}{|l|l|}
\hline
{\bf symbols} & {\bf use} \\
\hline
\verb@=@ & assignment \\
\verb@+= -= *= /= **= %= &= |= ^= &&= ||= <<= >>=@ & compound assignment \\
\verb@<=>@ & swap \\
\verb@..@ & range specifier \\
\verb@by@ & range/domain stride specifier \\
\verb@#@ & range count operator \\
\verb@...@ & variable argument lists \\
\verb@&& || ! & | ^ ~ << >>@ & logical/bitwise operators \\
\verb@== != <= >= < >@ & relational operators \\
\verb@+ - * / % **@ & arithmetic operators \\
\verb@:@ & type specifier \\
\verb@;@ & statement separator \\
\verb@,@ & expression separator \\
\verb@.@ & member access \\
\verb@?@ & type query \\
\verb@" '@ & string delimiters \\
\hline
\end{tabular}
\end{center}

\subsection{Grouping Tokens}
\label{Grouping_Tokens}
\index{lexical structure!braces}
\index{lexical structure!parentheses}
\index{lexical structure!brackets}

The following braces are part of the Chapel language:
\begin{center}
\begin{tabular}{|l|l|}
\hline
{\bf braces} & {\bf use} \\
\hline
\verb@( )@ & parenthesization, function calls, and tuples \\
\verb@[ ]@ & array literals, array types, forall expressions, and function calls \\
\verb@{ }@ & domain literals, block statements \\
\hline
\end{tabular}
\end{center}
