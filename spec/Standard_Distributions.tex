\sekshun{Standard Distributions}
\label{Standard_Distributions}
\index{distributions!standard}
\index{standard distributions (see also distributions, standard)}

The following table lists distributions standard to the Chapel
language:
\begin{center}
\begin{tabular}{|l|l|l|}
\hline
{\bf Distribution} & {\bf Module} & {\bf Supported Domain Types} \\
\hline
\chpl{block} & \chpl{BlockDist} & Rectangular \\
\chpl{cyclic} & \chpl{CyclicDist} & Rectangular \\
\chpl{block-cyclic} & \chpl{BlockCycDist} & Rectangular \\
\chpl{replicated} & \chpl{ReplicatedDist} & Rectangular \\
\hline
\end{tabular}
\end{center}

\begin{rationale}
Why supply any standard distributions?  A main design goal of Chapel
requires that the standard distributions be defined using the same
mechanisms available to Chapel programmers wishing to define their own
distributions or layouts~(\rsec{User_Defined_Domain_Maps}).  That way
there shouldn't be a necessary performance cost associated with
user-defined domain maps.  Nevertheless, distributions are an integral
part of the Chapel language which would feel incomplete without a good
set of standard distributions.  It is hoped that many distributions
will begin as user-defined domain maps and later become part of the
standard set of distributions.
\end{rationale}


\section{The Block Distribution}
\label{Block_Dist}
\index{Block distribution}
\index{distributions!standard!Block}

The standard Block distribution, defined in the
module \chpl{BlockDist}, maps indices to locales by partitioning the
indices into blocks according to a \emph{bounding box} argument.  It
is parameterized by the rank and index type of the domains it
supports.  Thus domains of different ranks or different index types
must be distributed with different Block distributions.

For Block distributions of rank $d$, given a bounding box
\begin{chapel}
[$l_1$..$h_1$, $\ldots$, $l_d$..$h_d$]
\end{chapel}
and an array of target locales defined over the domain
\begin{chapel}
[$0$..$n_1$-1, $\ldots$, $0$..$n_d-1$]
\end{chapel}
then a Block distribution maps an index $i$ to a locale by computing
the $k$\emph{th} component of an index $j$ into the array of target
locales from the $k$\emph{th} component of $i$ using the following
formula:
\[
j_k = \left\{
  \begin{array}{ll}
    0 & \mbox{if $i_k < l_k$} \\
    \left\lfloor\dfrac{n_k (i_k - l_k)}{h_k - l_k + 1}\right\rfloor & \mbox{if $i_k \geq l_k$ and $i_k \leq h_k$} \\
    n_k-1 & \mbox{if $i_k > h_k$} \\
  \end{array}
\right.
\]

The Block class constructor is defined as follows:
\begin{chapel}
proc Block(boundingBox: domain,
          targetLocales: [] locale = Locales, 
          dataParTasksPerLocale = $\mbox{{\it value in global config const of the same name}}$,
          dataParIgnoreRunningTasks = $\mbox{{\it value in global config const of the same name}}$,
          dataParMinGranularity = $\mbox{{\it value in global config const of the same name}}$,
          param rank = boundingBox.rank,
          type idxType = boundingBox.dim(1).eltType)
\end{chapel}

The argument \chpl{boundingBox} is a non-distributed domain defining a
bounding box used to partition the space of all indices across an
array of target locales.

The argument \chpl{targetLocales} is a non-distributed array
containing the target locales to which this distribution maps indices
and data.  The rank of \chpl{targetLocales} must match the rank of the
distribution, or be one.  If the rank of
\chpl{targetLocales} is one, a greedy heuristic is used to reshape the
array of target locales so that it matches the rank of the
distribution and each dimension contains an approximately equal number
of indices.

The
arguments \chpl{dataParTasksPerLocale}, \chpl{dataParIgnoreRunningTasks},
and \chpl{dataParMinGranularity} set the knobs that are used to
control intra-locale data parallelism for Block-distributed domains
and arrays in the same way that the global configuration constants of
these names control data parallelism for ranges and
default-distributed domains and arrays~\rsec{data_parallel_knobs}.

The \chpl{rank} and \chpl{idxType} arguments are inferred from the
\chpl{boundingBox} argument unless explicitly set.

\begin{example}
The following code declares a Block distribution with a bounding box
equal to the domain \chpl{Space} and declares an array, \chpl{A}, over
a domain declared over this distribution.  The computation in
the \chpl{forall} loop sets each array element to the ID of the locale
to which it is mapped.
\begin{chapel}
use BlockDist;

const Space = {1..8, 1..8};
const D: domain(2) dmapped Block(boundingBox=Space) = Space;
var A: [D] int;

forall a in A do
  a = a.locale.id;

writeln(A);
\end{chapel}
When run on 6 locales, the output is:
\begin{chapelprintoutput}{}
0 0 0 0 1 1 1 1
0 0 0 0 1 1 1 1
0 0 0 0 1 1 1 1
2 2 2 2 3 3 3 3
2 2 2 2 3 3 3 3
2 2 2 2 3 3 3 3
4 4 4 4 5 5 5 5
4 4 4 4 5 5 5 5
\end{chapelprintoutput}
\end{example}


\section{The Cyclic Distribution}
\label{Cyclic_Dist}
\index{Cyclic distribution}
\index{distributions!standard!Cyclic}
The standard Cyclic distribution, defined in the
module \chpl{CyclicDist}, maps indices to locales in a round-robin
pattern according to a \emph{start index} argument.  It is
parameterized by the rank and index type of the domains it supports.
Thus domains of different ranks or different index types must be
distributed with different Cyclic distributions.

For cyclic distributions of rank $d$, given a start index
\begin{chapel}
($s_1$, $\ldots$, $s_d$)
\end{chapel}
and an array of target locales defined over the domain
\begin{chapel}
[$0$..$n_1$-1, $\ldots$, $0$..$n_d$-1]
\end{chapel}
then a Cyclic distribution maps an index $i$ to a locale by computing
the $k$\emph{th} component of an index $j$ into the array of target
locales from the $k$\emph{th} component of $i$ using the following
formula:
\[j_k = i_k - s_k \pmod{n_k}\]

The Cyclic class constructor is defined as follows:
\begin{chapel}
proc Cyclic(startIdx,
           targetLocales: [] locale = Locales,
           dataParTasksPerLocale = $\mbox{{\it value in global config const of the same name}}$,
           dataParIgnoreRunningTasks = $\mbox{{\it value in global config const of the same name}}$,
           dataParMinGranularity = $\mbox{{\it value in global config const of the same name}}$,
           param rank: int = $\mbox{{\it rank inferred from startIdx}}$,
           type idxType = $\mbox{{\it index type inferred from startIdx}}$)
\end{chapel}

The argument \chpl{startIdx} is a tuple of integers defining an index that
will be distributed to the first locale in \chpl{targetLocales}. For a single
dimensional distribution \chpl{startIdx} can be an integer or a tuple with a
single element.

The argument \chpl{targetLocales} is a non-distributed array
containing the target locales to which this distribution maps indices
and data.  The rank of \chpl{targetLocales} must match the rank of the
distribution, or be one.  If the rank of
\chpl{targetLocales} is one, a greedy heuristic is used to reshape the
array of target locales so that it matches the rank of the
distribution and each dimension contains an approximately equal number
of indices.

The
arguments \chpl{dataParTasksPerLocale}, \chpl{dataParIgnoreRunningTasks},
and \chpl{dataParMinGranularity} set the knobs that are used to
control intra-locale data parallelism for Cyclic-distributed domains
and arrays in the same way that the global configuration constants of
these names control data parallelism for ranges and
default-distributed domains and arrays ~\rsec{data_parallel_knobs}.

The \chpl{rank} and \chpl{idxType} arguments are inferred from the
\chpl{startIdx} argument unless explicitly set.

\begin{example}
The following code declares a Cyclic distribution with a start index
of \chpl{(1,1)} and declares an array, \chpl{A}, over a domain
declared over this distribution.  The computation in the \chpl{forall}
loop sets each array element to the ID of the locale to which it is
mapped.
\begin{chapel}
use CyclicDist;

const Space = {1..8, 1..8};
const D: domain(2) dmapped Cyclic(startIdx=Space.low) = Space;
var A: [D] int;

forall a in A do
  a = a.locale.id;

writeln(A);
\end{chapel}
When run on 6 locales, the output is:
\begin{chapelprintoutput}{}
0 1 0 1 0 1 0 1
2 3 2 3 2 3 2 3
4 5 4 5 4 5 4 5
0 1 0 1 0 1 0 1
2 3 2 3 2 3 2 3
4 5 4 5 4 5 4 5
0 1 0 1 0 1 0 1
2 3 2 3 2 3 2 3
\end{chapelprintoutput}
\end{example}


\section{The Block-Cyclic Distribution}
\label{Block_Cyclic_Dist}
\index{Block-Cyclic distribution}
\index{distributions!standard!Block-Cyclic}
The standard Block-Cyclic distribution, defined in the
module \chpl{BlockCycDist}, maps blocks of indices to locales in a
round-robin pattern according to \emph{block size} and \emph{start
index} arguments.  It is parameterized by the rank and index type of
the domains it supports.  Thus domains of different ranks or different
index types must be distributed with different instances of the
BlockCyclic distributions.

For Block-Cyclic distributions of rank $d$, given a start index of
\begin{chapel}
($s_1$, $\ldots$, $s_d$)
\end{chapel}
and a block size of
\begin{chapel}
($b_1$, $\ldots$, $b_d$)
\end{chapel}
and an array of target locales defined over the domain
\begin{chapel}
[$0$..$n_1$-1, $\ldots$, $0$..$n_d$-1]
\end{chapel}
an index $i$ is mapped to a locale by computing
the $k$\emph{th} component of an index $j$ into the array of target
locales from the $k$\emph{th} component of $i$ using the following
formula:
\[j_k = ((i_k - s_k) / b_k) \pmod{n_k}\]

The Block-Cyclic class constructor is defined as follows:
\begin{chapel}
proc BlockCyclic(startIdx: rank*idxType,
                 blocksize: rank*int,
                 targetLocales: [] locale = Locales,
                 param rank: int = $\mbox{{\it rank inferred from startIdx}}$,
                 type idxType = $\mbox{{\it index type inferred from startIdx}}$)
\end{chapel}

The argument \chpl{startIdx} is a tuple of integers defining an index that
will be distributed to the first locale in \chpl{targetLocales}. For a single
dimensional distribution \chpl{startIdx} can be an integer or a tuple with a
single element.

The argument \chpl{blocksize} is a tuple of integers defining
the block size of indices that will be used in dealing out indices
to the locales.

The argument \chpl{targetLocales} is a non-distributed array
containing the target locales to which this distribution maps indices
and data.  The rank of \chpl{targetLocales} must match the rank of the
distribution, or be one.  If the rank of
\chpl{targetLocales} is one, a greedy heuristic is used to reshape the
array of target locales so that it matches the rank of the
distribution and each dimension contains an approximately equal number
of indices.

%The
%arguments \chpl{dataParTasksPerLocale}, \chpl{dataParIgnoreRunningTasks},
%and \chpl{dataParMinGranularity} set the knobs that are used to
%control intra-locale data parallelism for Block-Cyclic-distributed domains
%and arrays in the same way that the global configuration constants of
%these names control data parallelism for ranges and
%default-distributed domains and arrays ~\rsec{data_parallel_knobs}.

The \chpl{rank} and \chpl{idxType} arguments are inferred from the
\chpl{startIdx} argument unless explicitly set.

\begin{example}
The following code declares a Block-Cyclic distribution with a start index
of \chpl{(1,1)} and a blocksize of \chpl{(2,3)} and declares an array, \chpl{A}, over a domain
declared over this distribution.  The computation in the \chpl{forall}
loop sets each array element to the ID of the locale to which it is
mapped.
\begin{chapel}
use BlockCycDist;

const Space = {1..8, 1..8};
const D: domain(2) dmapped BlockCyclic(startIdx=Space.low, 
                                       blocksize=(2,3)) 
       = Space;
var A: [D] int;

forall a in A do
  a = a.locale.id;

writeln(A);
\end{chapel}
When run on 6 locales, the output is:
\begin{chapelprintoutput}{}
0 0 0 1 1 1 0 0
0 0 0 1 1 1 0 0
2 2 2 3 3 3 2 2
2 2 2 3 3 3 2 2
4 4 4 5 5 5 4 4
4 4 4 5 5 5 4 4
0 0 0 1 1 1 0 0
0 0 0 1 1 1 0 0
\end{chapelprintoutput}
\end{example}


\section{The Replicated Distribution}
\label{Replicated_Dist}
\index{Replicated distribution}
\index{distributions!standard!Replicated}

The standard Replicated distribution is defined in the module
\chpl{ReplicatedDist}.
It causes a domain and its arrays to be replicated across the locales.
% TODO: describe the option to replicate across only the given locales
% (and the "consistency" requirement, commented out below).
An array receives a distinct set of elements -- a \emph{replicand} --
allocated on each locale.
%
In other words, mapping a domain with the replicated distribution gives it
an implicit additional dimension -- over the locales,
making it behave as if there is one copy of its indices per locale.

Consistency among the replicands is \emph{not} preserved automatically.
That is, changes to one replicand of an array are never propagated to
the other replicands by the distribution implementation.

Replication is observable only in some cases, as described below.

\begin{future}
This behavior may change in the future. In particular,
we are considering changing it so that replication
is never observable. For example, only the local replicand would
be accessed in all cases.
\end{future}

Replication over locales is observable in the following cases:

\begin{itemize}
\item when iterating over a replicated domain or array
\item when printing them out with write() and similar functions
\item when zippering and the replicated domain/array is
  the first among the zippered items
\item when assigning into the replicated array
  (each replicand gets a copy)
\item when inquiring about the domain's \chpl{numIndices}
  or the array's \chpl{numElements}
%\item when accessing array element(s) from a locale that was not included
%  in the array passed explicitly to the ReplicatedDist constructor,
%  an out-of-bounds error will result
\end{itemize}

Replication is not observable, i.e., only the replicand
\emph{on the current locale} is accessed in the following cases:

\begin{itemize}
\item when examining certain domain properties:
  \chpl{dim(d)}, \chpl{dims()}, \chpl{low}, \chpl{high}, \chpl{stride};
  but not \chpl{numIndices}
\item when indexing into an array
\item when slicing an array
%  TODO: right?
\item when zippering and the first zippered item is not replicated
\item when assigning to a non-replicated array,
  i.e. the replicated array is on the right-hand side of the assignment
\item when there is only a single locale
 (trivially: there is only one replicand in this case)
\end{itemize}

E.g. when iterating, the number of iterations will be ({\it the number of
locales involved}) times ({\it the number of iterations over this domain if
it were distributed with the default distribution}). Note that serial iteration
visits the domain indices and array elements of all the replicands
\emph{from the current locale}.

% TODO
%Features/limitations:
%* Consistency/coherence among replicands' array elements is NOT maintained.
%* Only rectangular domains are presently supported.
%* Serial iteration over a replicated domain (or array) visits the indices
%  (or array elements) of all replicands *from the current locale*.
%* When replicating over user-provided array of locales, that array
%  must be "consistent" (see below).
%
%"Consistent" array requirement:
%* The array of locales passed to the ReplicatedDist constructor, if any,
%  must be "consistent".
%* A is "consistent" if for each ix in A.domain, A[ix].id == ix.
%* Tip: the desired set of locales does not correspond to a rectangular
%  (perhaps strided and/or multi-dimensional) domain, make the array's
%  domain associative over int.

\begin{example}
The following code declares a replicated domain, \chpl{Drepl}, and a
replicated array, \chpl{Arepl}, and accesses it in different ways:

\begin{chapel}
const Dbase = {1..5};
const Drepl: domain(1) dmapped ReplicatedDist() = Dbase;
var Abase: [Dbase] int;
var Arepl: [Drepl] int;

// only the current locale's replicand is accessed
Arepl[3] = 4;

// these iterate over Dbase, so
// only the current locale's replicand is accessed
forall (b,r) in zip(Abase,Arepl) b = r;
Abase = Arepl;

// these iterate over Drepl;
// each replicand of Arepl will be zippered against
// (and copied from) the entire Abase
forall (r,b) in zip(Arepl,Abase) r = b;
Arepl = Abase;

// sequential zippering will detect difference in sizes
// (if multiple locales)
for (b,r) in zip(Abase,Arepl) ... // error
for (r,b) in zip(Arepl,Abase) ... // error
\end{chapel}
\end{example}
