\sekshun{Task Parallelism and Synchronization}
\label{Task_Parallelism_and_Synchronization}

Chapel supports both task parallelism and data parallelism.  This
chapter details task parallelism as follows:
\begin{itemize}
\item \rsec{Task_parallelism} introduces tasks and task parallelism.
\item \rsec{Begin} describes the begin statement, an unstructured way
to introduce concurrency into a program.
\item \rsec{Synchronization_Variables} describes synchronization
variables, an unstructured mechanism for synchronizing tasks.
\item \rsec{Cobegin} describes the cobegin statement, a structured way to
introduce concurrency into a program.
\item \rsec{Coforall} describes the coforall loop, another structured way to
introduce concurrency into a program.
\item \rsec{Sync_Statement} describes the sync statement, a structured
way to control parallelism.
\item \rsec{Serial} describes the serial statement, a structured way to suppress
parallelism.
\item \rsec{Atomic_Statement} describes the atomic statement, a construct to
support atomic transactions.
\end{itemize}

\section{Tasks and Task Parallelism}
\label{Task_parallelism}

A Chapel \emph{task} is a distinct context of execution that maybe
running concurrently with other tasks.  Chapel provides a simple
construct, the begin statement, to create tasks, introducing
concurrency into a program in an unstructured way.  In addition,
Chapel introduces two type qualifiers, \chpl{sync} and \chpl{single},
for synchronization between tasks.

Chapel provides two constructs, the cobegin and coforall statements,
to introduce concurrency in a more structured way.  These constructs
create multiple tasks but do not continue until the tasks have
completed.  In addition, Chapel provides two constructs, the sync- and
serial statements, to suppress parallelism and insert synchronization.
All four of these constructs can be implemented through judicious uses
of the unstructured task-parallel constructs described in the previous
section.


\section{The Begin Statement}
\label{Begin}
\index{begin@\chpl{begin}}

The begin statement creates a task to execute a statement.  The syntax
for the begin statement is given by
\begin{syntax}
begin-statement:
  `begin' statement
\end{syntax}
Control continues concurrently with the statement following the
begin statement.

\begin{chapelexample}{beginUnordered.chpl}
The code
\begin{chapel}
begin writeln("output from spawned task");
writeln("output from main task");
\end{chapel}
\begin{chapelprediff}
\#!/usr/bin/env sh
testname=$1
outfile=$2
sort $outfile > $outfile.2
mv $outfile.2 $outfile
\end{chapelprediff}
\begin{chapeloutput}
output from main task
output from spawned task
\end{chapeloutput}
executes two \chpl{writeln} statements that output the strings to the
terminal, but the ordering is purposely unspecified.  There is no
guarantee as to which statement will execute first.  When the
begin statement is executed, a new task is created that will execute
the \chpl{writeln} statement within it.  However, execution will
continue immediately after task creation with the next statement.
\end{chapelexample}

Yield and return statements are not allowed in begin blocks.  Break
and continue statements may not be used to exit a begin block.

%
% TODO: Future environment about task teams.
%

\section{Synchronization Variables}
\label{Synchronization_Variables}
\index{synchronization variables!sync@\chpl{sync}}
\index{synchronization variables!sync@\chpl{single}}
\index{sync@\chpl{sync}}
\index{single@\chpl{single}}

Synchronization variables have a logical state associated with the
value.  The state of the variable is either {\em full} or {\em empty}.
Normal reads of a synchronization variable cannot proceed until the
variable's state is full.  Normal writes of a synchronization variable
cannot proceed until the variable's state is empty.

Chapel supports two types of synchronization variables: sync and
single.  Both types behave similarly, expect that a single variable
may only be written once.  Consequently, when a sync variable is read,
its state transitions to empty, where as when a single variable is
read, its state does not change.  When either type of synchronization
variable is written, its state transitions to full.

Sync and single are type qualifiers and preceed the variable's type in
the declaration.  Sync and single are only supported for all Chapel
primitive types (~\rsec{Primitive_Types}) except strings and complex.

If a task attempts to read or write a synchronization variable that is
not in the correct state, the task is suspended.  When the variable
transitions to the correct state, the task is resumed.  If there are
multiple tasks blocked waiting for the state transition, one is
non-deterministically selected to proceed and the others continue to
wait.

A synchronization variable is specified with a sync or single type
given by the following syntax:
\begin{syntax}
sync-type:
  `sync' type-specifier

single-type:
  `single' type-specifier
\end{syntax}

If a synchronization variable declaration has an initialization
expression, then the variable is initially full, otherwise it is
initially empty.

\begin{chapelexample}{beginWithSyncVar.chpl}
The code
\begin{chapel}
class Tree {
  var isLeaf: bool;
  var left, right: Tree;
  var value: int;

  proc sum():int {
    if (isLeaf) then 
       return value;

    var x(*\texttt{\$}*): sync int;
    begin x(*\texttt{\$}*) = left.sum();
    var y = right.sum();
    return x(*\texttt{\$}*) + y;
  }
}
\end{chapel}
\begin{chapelpost}
var tree: Tree = new Tree(false, new Tree(false, new Tree(true, nil, nil, 1),
                                                 new Tree(true, nil, nil, 1), 1),
                                 new Tree(false, new Tree(true, nil, nil, 1),
                                                 new Tree(true, nil, nil, 1), 1), 1);
writeln(tree.sum());
\end{chapelpost}
\begin{chapeloutput}
4
\end{chapeloutput}
the sync variable \chpl{x$\mbox{\texttt{\$}}$} is assigned by an
asynchronous task created with the begin statement.  The task
returning the sum waits on the reading of \chpl{x$\mbox{\texttt{\$}}$}
until it has been assigned.  By convention, synchronization variables
end in \texttt{\$} to provide a visual cue to the programmer
indicating that the task may block.
\end{chapelexample}

\begin{chapelexample}{syncCounter.chpl}
Sync variables are useful for tallying data from multiple tasks as
well.  A sync variable of type \chpl{int} is read and then written
during an update so the full-empty semantics make these updates
atomic.  The code
\begin{chapel}
var count(*\texttt{\$}*): sync int = 0;
begin count(*\texttt{\$}*) = count(*\texttt{\$}*) + 1;
begin count(*\texttt{\$}*) = count(*\texttt{\$}*) + 1;
begin count(*\texttt{\$}*) = count(*\texttt{\$}*) + 1;
\end{chapel}
\begin{chapelpost}
while count(*\texttt{\$}*).readFF() != 3 do ;
writeln("count is: ", count(*\texttt{\$}*));
\end{chapelpost}
\begin{chapeloutput}
count is: 3
\end{chapeloutput}
creates three tasks to increment \chpl{count$\mbox{\texttt{\$}}$}.
If \chpl{count$\mbox{\texttt{\$}}$} was not a sync variable, this code
would be unsafe because between the points at which one task
reads \chpl{count$\mbox{\texttt{\$}}$} and
writes \chpl{count$\mbox{\texttt{\$}}$}, another task may increment
it.
\end{chapelexample}

\begin{chapelexample}{singleVar.chpl}
The following code implements a simple split-phase barrier using a
single variable.
\begin{chapelpre}
config const n = 44;
proc work(i) {
  // do nothing
}
\end{chapelpre}
\begin{chapel}
var count(*\texttt{\$}*): sync int = n;  // counter which also serves as a lock
var release(*\texttt{\$}*): single bool; // barrier release

forall t in 1..n do begin {
  work(t);
  var myc = count(*\texttt{\$}*);  // read the count, set state to empty
  if myc!=1 {
    write(".");
    count(*\texttt{\$}*) = myc-1;  // update the count, set state to full
    // we could also do some work here before blocking
    release(*\texttt{\$}*);
  } else {
    release(*\texttt{\$}*) = true;  // last one here, release everyone
    writeln("done");
  }
}
\end{chapel}
\begin{chapeloutput}
...........................................done
\end{chapeloutput}
In each iteration of the forall loop after the work is completed, the
task reads the \chpl{count$\mbox{\texttt{\$}}$} variable, which is
used to tally the number of tasks that have arrived.  All tasks except
the last task to arrive will block while trying to read the
variable \chpl{release$\mbox{\texttt{\$}}$}.  The last task to arrive
will write to \chpl{release$\mbox{\texttt{\$}}$}, setting its state to
full at which time all the other tasks can be unblocked and run.
\end{chapelexample}

\index{synchronization types!formal arguments}
If a formal argument is a synchronization type, the actual is passed
by reference and the argument itself is a valid lvalue.  The
unqualified types \chpl{sync} or \chpl{single} can also be used to
specify a generic formal argument.  In this case, the actual must be a
sync or single variable and it is passed by reference.

For generic formal arguments with unspecified types
(\rsec{Formal_Arguments_of_Generic_Type}), an actual that
is \chpl{sync} or \chpl{single} is read before being passed to the
function and the generic formal argument's type is set to the base
type of the actual.


\subsection{Predefined Single and Sync Methods}
\label{Functions_on_Synchronization_Variables}
\index{synchronization variables!predefined methods on}

The following methods are defined for variables of sync and single
type.

\index{readFE@\chpl{readFE}}
\begin{protohead}
proc (sync t).readFE(): t
\end{protohead}
\begin{protobody}
Returns the value of the sync variable.  This method blocks until the
sync variable is full.  The state of the sync variable is set to empty
when this method completes.
\end{protobody}

\index{readFF@\chpl{readFF}}
\begin{protohead}
proc (sync t).readFF(): t
proc (single t).readFF(): t
\end{protohead}
\begin{protobody}
Returns the value of the sync or single variable.  This method blocks
until the sync or single variable is full.  The state of the sync or
single variable remains full when this method completes.
\end{protobody}

\index{readXX@\chpl{readXX}}
\begin{protohead}
proc (sync t).readXX(): t
proc (single t).readXX(): t
\end{protohead}
\begin{protobody}
Returns the value of the sync or single variable.  This method is non-blocking
and the state of the sync or single variable is unchanged when this method
completes.
\end{protobody}

\index{writeEF@\chpl{writeEF}}
\begin{protohead}
proc (sync t).writeEF(v: t)
proc (single t).writeEF(v: t)
\end{protohead}
\begin{protobody}
Assigns \chpl{v} to the value of the sync or single variable.  This
method blocks until the sync or single variable is empty.  The state
of the sync or single variable is set to full when this method
completes.
\end{protobody}

\index{writeFF@\chpl{writeFF}}
\begin{protohead}
proc (sync t).writeFF(v: t)
\end{protohead}
\begin{protobody}
Assigns \chpl{v} to the value of the sync variable.  This method
blocks until the sync variable is full.  The state of the sync
variable remains full when this method completes.
\end{protobody}

\index{writeXF@\chpl{writeXF}}
\begin{protohead}
proc (sync t).writeXF(v: t)
\end{protohead}
\begin{protobody}
Assigns \chpl{v} to the value of the sync variable.  This method is
non-blocking and the state of the sync variable is set to full when
this method completes.
\end{protobody}

\index{reset@\chpl{reset}}
\begin{protohead}
proc (sync t).reset()
\end{protohead}
\begin{protobody}
Assigns the default value of type \chpl{t} to the value of the sync
variable.  This method is non-blocking and the state of the sync
variable is set to empty when this method completes.
\end{protobody}

\index{isFull@\chpl{isFull}}
\begin{protohead}
proc (sync t).isFull: bool
proc (single t).isFull: bool
\end{protohead}
\begin{protobody}
Returns \chpl{true} if the sync or single variable is full and \chpl{false}
otherwise.  This method is non-blocking and the state of the sync or single
variable is unchanged when this method completes.
\end{protobody}

Note that \chpl{writeEF} and \chpl{readFE}/\chpl{readFF} methods
(for \chpl{sync} and \chpl{single} variables, respectively) are
implicitly invoked for writes and reads of synchronization variables.


\begin{chapelexample}{syncMethods.chpl}
Given the following declarations
\begin{chapel}
{ // }
var x(*\texttt{\$}*): sync int;
var y(*\texttt{\$}*): single int;
var z: int;
\end{chapel}
the code
\begin{chapel}
x(*\texttt{\$}*) = 5;
y(*\texttt{\$}*) = 6;
z = x(*\texttt{\$}*) + y(*\texttt{\$}*);
\end{chapel}
\begin{chapelnoprint}
writeln((x(*\texttt{\$}*).readXX(), y(*\texttt{\$}*), z));
// {
}
{ // }
var x(*\texttt{\$}*): sync int;
var y(*\texttt{\$}*): single int;
var z: int;
\end{chapelnoprint}
is equivalent to
\begin{chapel}
x(*\texttt{\$}*).writeEF(5);
y(*\texttt{\$}*).writeEF(6);
z = x(*\texttt{\$}*).readFE() + y(*\texttt{\$}*).readFF();
\end{chapel}
\begin{chapelpost}
writeln((x(*\texttt{\$}*).readXX(), y(*\texttt{\$}*), z));
// {
}
\end{chapelpost}
\begin{chapeloutput}
(5, 6, 11)
(5, 6, 11)
\end{chapeloutput}
\end{chapelexample}

\section{The Cobegin Statement}
\label{Cobegin}
\index{cobegin@\chpl{cobegin}}

The cobegin statement is used to introduce concurrency within a
block.  The \chpl{cobegin} statement syntax is
\begin{syntax}
cobegin-statement:
  `cobegin' block-statement
\end{syntax}
A new task is created for each statement in the block.  Control
continues when all of the tasks have finished.

Return statements are not allowed in cobegin blocks.  Yield statement
may only be lexically enclosed in cobegin blocks in parallel
iterators~(\rsec{Parallel_Iterators}).  Break and continue statements
may not be used to exit a cobegin block.


\begin{chapelexample}{cobeginAndEquivalent.chpl}
The cobegin statement
\begin{chapelpre}
var s1, s2: sync int;
proc stmt1() { s1; }
proc stmt2() { s2; s1 = 1; }
proc stmt3() { s2 = 1; }
\end{chapelpre}
\begin{chapel}
cobegin {
  stmt1();
  stmt2();
  stmt3();
}
\end{chapel}
is equivalent to the following code that uses only begin statements
and single variables to introduce concurrency and synchronize:
\begin{chapel}
var s1(*\texttt{\$}*), s2(*\texttt{\$}*), s3(*\texttt{\$}*): single bool;
begin { stmt1(); s1(*\texttt{\$}*) = true; }
begin { stmt2(); s2(*\texttt{\$}*) = true; }
begin { stmt3(); s3(*\texttt{\$}*) = true; }
s1(*\texttt{\$}*); s2(*\texttt{\$}*); s3(*\texttt{\$}*);
\end{chapel}
\begin{chapeloutput}
\end{chapeloutput}
Each begin statement is executed concurrently but control does not
continue past the final line above until each of the single variables
is written, thereby ensuring that each of the functions has finished.
\end{chapelexample}

\section{The Coforall Loop}
\label{Coforall}
\index{coforall@\chpl{coforall}}
\index{coforall loops}

The coforall loop is a variant of the cobegin statement in loop form.
The syntax for the coforall loop is given by
\begin{syntax}
coforall-statement:
  `coforall' index-var-declaration `in' iteratable-expression `do' statement
  `coforall' index-var-declaration `in' iteratable-expression block-statement
  `coforall' iteratable-expression `do' statement
  `coforall' iteratable-expression block-statement
\end{syntax}

The \chpl{coforall} loop creates a separate task for each iteration of
the loop.  Control continues with the statement following
the \chpl{coforall} loop after all tasks corresponding to the
iterations of the loop have completed.

Return statements are not allowed in coforall blocks.  Yield statement
may only be lexically enclosed in coforall blocks in parallel
iterators~(\rsec{Parallel_Iterators}).  Break and continue statements
may not be used to exit a coforall block.

\begin{chapelexample}{coforallAndEquivalent.chpl}
The coforall statement
\begin{chapelpre}
iter iterator() { for i in 1..3 do yield i; }
proc body() { }
\end{chapelpre}
\begin{chapel}
coforall i in iterator() {
  body();
}
\end{chapel}
is equivalent to the following code that uses only begin statements
and sync and single variables to introduce concurrency and
synchronize:
\begin{chapel}
var runningCount(*\texttt{\$}*): sync int = 1;
var finished(*\texttt{\$}*): single bool;
for i in iterator() {
  runningCount(*\texttt{\$}*) += 1;
  begin {
    body();
    var tmp = runningCount(*\texttt{\$}*);
    runningCount(*\texttt{\$}*) = tmp-1;
    if tmp == 1 then finished(*\texttt{\$}*) = true;
  }
}
var tmp = runningCount(*\texttt{\$}*);
runningCount(*\texttt{\$}*) = tmp-1;
if tmp == 1 then finished(*\texttt{\$}*) = true;
finished(*\texttt{\$}*);
\end{chapel}
\begin{chapeloutput}
\end{chapeloutput}
Each call to \chpl{body()} executes concurrently because it is in a
begin statement.  The sync
variable \chpl{runningCount$\mbox{\texttt{\$}}$} is used to keep track
of the number of executing tasks plus one for the main task.  When
this variable reaches zero, the single
variable \chpl{finished$\mbox{\texttt{\$}}$} is used to signal that
all of the tasks have completed.  Thus control does not continue past
the last line until all of the tasks have completed.
\end{chapelexample}

\section{The Sync Statement}
\label{Sync_Statement}
\index{sync@\chpl{sync}}

The sync statement acts as a join of all dynamically encountered
begins from within a statement.  The syntax for the sync statement is
given by
\begin{syntax}
sync-statement:
  `sync' statement
  `sync' block-statement
\end{syntax}

Return statements are not allowed in sync statement blocks.  Yield
statement may only be lexically enclosed in sync statement blocks in
parallel iterators~(\rsec{Parallel_Iterators}).  Break and continue
statements may not be used to exit a sync statement block.

\begin{chapelexample}{syncStmt1.chpl}
The sync statement can be used to wait for many dynamically created
tasks.
\begin{chapelpre}
config const n = 9;
proc work() {
  write(".");
}
\end{chapelpre}
\begin{chapel}
sync for i in 1..n do begin work();
\end{chapel}
\begin{chapelpost}
writeln("done");
\end{chapelpost}
\begin{chapeloutput}
.........done
\end{chapeloutput}
The for loop is within a sync statement and thus the tasks created
in each iteration of the loop must complete before the continuing past
the sync statement.
\end{chapelexample}

\begin{chapelexample}{syncStmt2.chpl}
The sync statement
\begin{chapelpre}
proc stmt1() { }
proc stmt2() { }
\end{chapelpre}
\begin{chapel}
sync {
  begin stmt1();
  begin stmt2();
}
\end{chapel}
is similar to the following cobegin statement
\begin{chapel}
cobegin {
  stmt1();
  stmt2();
}
\end{chapel}
\begin{chapeloutput}
\end{chapeloutput}
except that if begin statements are dynamically encountered
when \chpl{stmt1()} or \chpl{stmt2()} are executed, then the former
code will wait for these begin statements to complete whereas the
latter code will not.
\end{chapelexample}

\section{The Serial Statement}
\label{Serial}
\index{serial@\chpl{serial}}

The \chpl{serial} statement can be used to dynamically disable
parallelism.  The syntax is:
\begin{syntax}
serial-statement:
  `serial' expression `do' statement
  `serial' expression block-statement
\end{syntax}
where the expression evaluates to a bool type.  Independent of that
value, the \sntx{statement} is evaluated. If the expression is true,
any dynamically encountered code that would result in new tasks is
executed without creating any new tasks.  In effect, execution is
serialized.

\begin{chapelexample}{serialStmt1.chpl}
In the code
\begin{chapelpre}
config const lo = 9;
config const hi = 23;
proc work(i) {
  if \_\_primitive("task\_get\_serial") then
    writeln("serial ", i);
}
\end{chapelpre}
\begin{chapel}
proc f(i) {
  serial i<13 {
    cobegin {
      work(i);
      work(i);
    }
  }
}

for i in lo..hi {
  f(i);
}
\end{chapel}
\begin{chapelpost}
\end{chapelpost}
\begin{chapeloutput}
serial 9
serial 9
serial 10
serial 10
serial 11
serial 11
serial 12
serial 12
\end{chapeloutput}
the serial statement in procedure f() inhibits concurrent execution of
work() if the variable i is less than 13.
\end{chapelexample}

\begin{chapelexample}{serialStmt2.chpl}
The code
\begin{chapelpre}
proc stmt1() { write(1); }
proc stmt2() { write(2); }
proc stmt3() { write(3); }
proc stmt4() { write(4); }
var n = 3;
\end{chapelpre}
\begin{chapel}
serial true {
  begin stmt1();
  cobegin {
    stmt2();
    stmt3();
  }
  coforall i in 1..n do stmt4();
}
\end{chapel}
is equivalent to
\begin{chapel}
stmt1();
{
  stmt2();
  stmt3();
}
for i in 1..n do stmt4();
\end{chapel}
\begin{chapelpost}
writeln();
\end{chapelpost}
\begin{chapeloutput}
123444123444
\end{chapeloutput}
because the expression evaluated to determine whether to serialize
always evaluates to true.
\end{chapelexample}

\section{Atomic Statements}
\label{Atomic_Statement}
\index{atomic transactions}
\index{atomic statement}
\index{atomic@\chpl{atomic}}

\begin{openissue}
  This section describes a feature that is a work-in-progress.  We seek feedback
  and collaboration in this area from the broader community.
\end{openissue}

The \emph{atomic statement} is used to specify that a statement should appear
to execute atomically from other tasks' point of view.
In particular, no task will see memory in a state that would reflect that
the atomic statement had begun executing but had not yet completed.

\begin{openissue}
  This definition of the atomic statement provides a notion of {\em
    strong atomicity} since the action will appear atomic to any task
  at any point in its execution.  For performance reasons, it could be
  more practical to support {\em weak atomicity} in which the
  statement's atomicity is only guaranteed with respect to other
  atomic statements.  We may also pursue using atomic type qualifiers
  as a means of marking data that should be accessed atomically inside
  or outside an atomic section.
\end{openissue}

The syntax for the atomic statement is given by:
\begin{syntax}
atomic-statement:
  `atomic' statement
\end{syntax}

%\begin{chapelexample}{atomicStmt}
\begin{example}
The following code illustrates the use of an atomic statement
to perform an insertion into a doubly-linked list:

\begin{chapelpre}
class Node {
  var data: int;
  var next: Node;
  var prev: Node;
}
var head = new Node(1);
head.insertAfter(new Node(4));
head.insertAfter(new Node(2));

var obj = new Node(3);
head.next.insertAfter(obj);
\end{chapelpre}
\begin{chapel}
proc Node.insertAfter(newNode: Node) {
  atomic {
    newNode.prev = this;
    newNode.next = this.next;
    if this.next then this.next.prev = newNode;
    this.next = newNode;
  }
}
\end{chapel}
\begin{chapelpost}
writeln(head.data, head.next.data, head.next.next.data, head.next.next.next.data);
\end{chapelpost}
\begin{chapeloutput}
atomic.chpl:13: warning: atomic keyword is ignored (not implemented)
1234
\end{chapeloutput}
The use of the atomic statement in this routine prevents other tasks
from viewing the list in a partially-updated state in which the
pointers might not be self-consistent.
\end{example}
%\end{chapelexample}
