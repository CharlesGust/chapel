\sekshun{Tuples}
\label{Tuples}
\index{tuples}

A tuple is an ordered set of components that allows for the
specification of a light-weight collection of values.  As the examples
in this chapter illustrate, tuples are a boon to the Chapel
programmer.  In addition to making it easy to return multiple values
from a function, tuples help to support multidimensional indices, to
group arguments to functions, and to specify mathematical concepts.

\subsection{Tuple Types}
\label{Tuple_Types}
\index{tuples!types}

A tuple type is defined by a fixed number (a compile-time constant) of
component types.  It can be specified by a parenthesized,
comma-separated list of types.  The number of types in the list
defines the size of the tuple; the types themselves specify the
component types.

The syntax of a tuple type is given by:
\begin{syntax}
tuple-type:
  ( type-specifier , type-list )

type-list:
  type-specifier
  type-specifier , type-list
\end{syntax}

\index{tuples!homogeneous}
\index{* (tuples)@\chpl{*} tuples}
A homogeneous tuple is a special-case of a general tuple where the
types of the components are identical.  Homogeneous tuples have fewer
restrictions for how they can be indexed~(\rsec{Tuple_Indexing}).
Homogeneous tuple types can be defined using the above syntax, or they
can be defined as a product of an integral parameter (a compile-time
constant integer) and a type.  This latter specification is
implemented by overloading \chpl{*} with the following prototype:
\begin{chapel}
def *(param p: int, type t) type
\end{chapel}

\begin{rationale}
Homogeneous tuples require the size to be specified as a parameter
(compile-time constant).  This avoids any overhead associated with
storing the runtime size in the tuple.  It also avoids the question as
to whether a non-parameter size should be part of the type of the
tuple.  If a programmer requires a non-parameter value to define a
data structure, an array may be a better choice.
\end{rationale}

\begin{example}
\begin{chapelpre}
\end{chapelpre}
The statement
\begin{chapel}
var x1: (string, real),
    x2: (int, int, int),
    x3: 3*int;
\end{chapel}
defines three variables.  Variable \chpl{x1} is a 2-tuple with
component types \chpl{string} and \chpl{real}.  Variables \chpl{x2}
and \chpl{x3} are homogeneous 3-tuples with component type \chpl{int}.
The types of \chpl{x2} and \chpl{x3} are identical even though they
are specified in different ways.
\begin{chapelpost}
writeln((x1,x2,x3));
\end{chapelpost}
\begin{chapeloutput}
((, 0.0), (0, 0, 0), (0, 0, 0))
\end{chapeloutput}
\end{example}

Note that if a single type is delimited by parentheses, the
parentheses only impact precedence.  Thus \chpl{(int)} is equivalent
to \chpl{int}.  Nevertheless, tuple types with a single component type
are legal and useful.  One way to specify a 1-tuple is to use the
overloaded \chpl{*} operator since every 1-tuple is trivially a
homogeneous tuple.

\begin{rationale}
Like parentheses around expressions, parentheses around types are
necessary for grouping in order to avoid the default precedence of the
grammar.  Thus it is not the case that we would always want to create
a tuple.  The type \chpl{3*(3*int)} specifies a 3-tuple of 3-tuples of
integers rather than a 3-tuple of 1-tuples of 3-tuples of integers.
The type \chpl{3*3*int}, on the other hand, specifies a 9-tuple of
integers.
\end{rationale}

\subsection{Tuple Values}
\label{Tuple_Values}
\index{tuples!values}

A value of a tuple type attaches a value to each component type.
Tuple values can be specified by a parenthesized, comma-separated list
of expressions.  The number of expressions in the list defines the
size of the tuple; the types of these expressions specify the
component types of the tuple.

The syntax of a tuple expression is given by:
\begin{syntax}
tuple-expression:
  ( expression , expression-list )

expression-list:
  expression
  expression , expression-list
\end{syntax}

\begin{example}
\begin{chapelpre}
\end{chapelpre}
The statement
\begin{chapel}
var x1: (string, real) = ("hello", 3.14),
    x2: (int, int, int) = (1, 2, 3),
    x3: 3*int = (4, 5, 6);
\end{chapel}
defines three variables.  Variable \chpl{x1} is a 2-tuple with
component types \chpl{string} and \chpl{real}.  It is initialized such
that the first component is \chpl{"hello"} and the second
component is \chpl{3.14}.  Variables \chpl{x2} and \chpl{x3} are
homogeneous 3-tuples with component type \chpl{int}.  Their
initialization expressions specify 3-tuples of integers.
\begin{chapelpost}
writeln((x1,x2,x3));
\end{chapelpost}
\begin{chapeloutput}
((hello, 3.14), (1, 2, 3), (4, 5, 6))
\end{chapeloutput}
\end{example}

\index{tuple@\chpl{tuple}}
The function
\begin{chapel}
def tuple(x...) return x;
\end{chapel}
is defined in the standard context to create arbitrary tuples.

\begin{example}
\begin{chapelpre}
\end{chapelpre}
The statement
\begin{chapel}
var x1 =      ("hello", 3.14),
    x2 = tuple("hello", 3.14),
    x3 = tuple(1);
\end{chapel}
creates two identical tuples \chpl{x1} and \chpl{x2}, and a 1-tuple of
an integer \chpl{x3} with its component initialized to \chpl{1}.
\begin{chapelpost}
writeln((x1,x2,x3));
\end{chapelpost}
\begin{chapeloutput}
((hello, 3.14), (hello, 3.14), (1))
\end{chapeloutput}
\end{example}

Note that if a single expression is delimited by parentheses, the
parentheses only impact precedence.  Thus \chpl{(1)} is equivalent
to \chpl{1}.  Tuple expressions with a single component are legal and
useful.  As seen in the example above, one way to specify a 1-tuple is
to use the standard \chpl{tuple} function.

Tuple expressions are evaluated similarly to function calls where the
arguments are all generic with no explicit intent.  So a tuple
expression containing an array does not copy the array.  The semantics
regarding passing tuples to functions is forthcoming.

\subsection{Tuple Assignment}
\label{Tuple_Assignment}
\index{assignment!tuples}
\index{tuples!assignment}

In tuple assignment, the components of the tuple on the left-hand side
of the assignment operator are each assigned the components of the
tuple on the right-hand side of the assignment.  These assignments
occur in component order (component one followed by component two,
etc.).

\subsection{Tuple Destructuring}
\label{Tuple_Destructuring}
\index{tuples!destructuring}

Tuples can be split into their components in four ways:
\begin{itemize}
\item In assignment where multiple expression on the left-hand side of
the assignment operator are grouped using tuple notation.
\item In variable declarations where multiple variables in a
declaration are grouped using tuple notation.
\item In for, forall, and coforall loops (statements and expressions)
where multiple indices in a loop are grouped using tuple notation.
\item In function calls where multiple formal arguments in a function
declaration are grouped using tuple notation.
\item In an expression context that accepts a comma-separated list of
expressions where a tuple expression is expanded in place using the
tuple expansion expression.
\end{itemize}

\subsubsection{Splitting a Tuple with Assignment}

When multiple expression on the left-hand side of an assignment
operator are grouped using tuple notation, the tuple on the right-hand
side is split into its components.  The number of grouped expressions
must be equal to the size of the tuple on the right-hand side.  In
addition to the usual assignment evaluation order of left to right,
the assignment is evaluated in component order.

\begin{example}
\begin{chapelpre}
\end{chapelpre}
The code
\begin{chapel}
var a, b, c: int;
(a, (b, c)) = (1, (2, 3));
\end{chapel}
defines three integer variables \chpl{a}, \chpl{b}, and \chpl{c}.  The
second line then splits the tuple \chpl{(1, (2, 3))} such that \chpl{1}
is assigned to \chpl{a}, \chpl{2} is assigned to \chpl{b},
and \chpl{3} is assigned to \chpl{c}.
\begin{chapelpost}
writeln((a, b, c));
\end{chapelpost}
\begin{chapeloutput}
(1, 2, 3)
\end{chapeloutput}
\end{example}

\begin{example}
\begin{chapelpre}
\end{chapelpre}
The code
\begin{chapel}
var A = [i in 1..4] i;
writeln(A);
(A(1..2), A(3..4)) = (A(3..4), A(1..2));
writeln(A);
\end{chapel}
creates a non-distributed, one-dimensional array containing the four
integers from \chpl{1} to \chpl{4}.  Line 2 outputs \chpl{1 2 3 4}.
Line 3 does what appears to be a swap of array slices.  However,
because the tuple is created with array aliases (like a function
call), the assignment to the second component uses the values just
overwritten in the assignment to the first component.  Line 4
outputs \chpl{3 4 3 4}.
\begin{chapelpost}
\end{chapelpost}
\begin{chapeloutput}
1 2 3 4
3 4 3 4
\end{chapeloutput}
\end{example}

\index{tuples!eliding components}
When splitting a tuple with assignment, the expressions that are
grouped using the tuple notation may be omitted.  In this case, the
expression on the right-hand side of the assignment operator is
evaluated, but its value is not assigned.

\begin{example}
\begin{chapelpre}
\end{chapelpre}
The code
\begin{chapel}
def f()
  return (1, 2);

var x: int;
(x, ) = f();
\end{chapel}
defines a function that returns a 2-tuple, declares an integer
variable \chpl{x}, calls the function, assigns the first component in
the returned tuple to \chpl{x}, and ignores the second component in
the returned tuple.  The value of \chpl{x} becomes \chpl{1}.
\begin{chapelpost}
writeln(x);
\end{chapelpost}
\begin{chapeloutput}
1
\end{chapeloutput}
\end{example}

\subsubsection{Splitting a Tuple into Multiple Variables}
\label{Variable_Declarations_in_a_Tuple}
\index{tuples!variable declarations grouped as}

When multiple variables in a declaration are grouped using tuple
notation, the tuple type and/or tuple initialization expression are
split into their components.  The number of grouped variables must be
equal to the size of the tuple type and/or tuple initialization
expression.  The variables are initialized in component order.

The syntax of grouped variable declarations is defined
in~\rsec{Variable_Declarations}.

\begin{example}
\begin{chapelpre}
\end{chapelpre}
The code
\begin{chapel}
var (a, (b, c)) = (1, (2, 3));
\end{chapel}
defines three integer variables \chpl{a}, \chpl{b}, and \chpl{c}.  It
splits the tuple \chpl{(1, (2, 3))} such that \chpl{1}
initializes \chpl{a}, \chpl{2} initializes \chpl{b}, and \chpl{3}
initializes \chpl{c}.
\begin{chapelpost}
writeln((a, b, c));
\end{chapelpost}
\begin{chapeloutput}
(1, 2, 3)
\end{chapeloutput}
\end{example}

Grouping variable declarations using tuple notation allows a 1-tuple
to be destructured by enclosing a single variable declaration in
parentheses.
\begin{example}
\begin{chapelpre}
\end{chapelpre}
The code
\begin{chapel}
var (a) = tuple(1);
\end{chapel}
initialize the new variable \chpl{a} to 1.
\begin{chapelpost}
writeln(a);
\end{chapelpost}
\begin{chapeloutput}
1
\end{chapeloutput}
\end{example}

\index{tuples!eliding components}
When splitting a tuple into multiple variable declarations, the
variables that are grouped using the tuple notation may be omitted.  In
this case, a variable is not defined for any omitted component.

\begin{example}
\begin{chapelpre}
\end{chapelpre}
The code
\begin{chapel}
def f()
  return (1, 2);

var (x, ) = f();
\end{chapel}
defines a function that returns a 2-tuple, calls the function,
declares and initializes variable \chpl{x} to the first component in
the returned tuple, and ignores the second component in the returned
tuple.  The value of \chpl{x} is initialized to \chpl{1}.
\begin{chapelpost}
writeln(x);
\end{chapelpost}
\begin{chapeloutput}
1
\end{chapeloutput}
\end{example}

\subsubsection{Splitting a Tuple into Mutiple Indices}
\label{Indices_in_a_Tuple}
\index{tuples!indices grouped as}

When multiple indices in a loop are grouped using tuple notation, the
values returned by the iterator are split into their components.  The
number of grouped indices must be equal to the size of the tuple
returned by the iterator.

\begin{example}
\begin{chapelpre}
\end{chapelpre}
The code
\begin{chapel}
def bar() {
  yield (1, 1);
  yield (2, 2);
}

for (i,j) in bar() do
  writeln(i+j);
\end{chapel}
defines a simple iterator that yields two 2-tuples before completing.
The for-loop uses a tuple notation to group two indices that take
their values from the iterator.
\begin{chapelpost}
\end{chapelpost}
\begin{chapeloutput}
2
4
\end{chapeloutput}
\end{example}

\index{tuples!eliding components}
When splitting a tuple into multiple indices, the indices that are
grouped using the tuple notation may be omitted.  In this case, a new
index is not defined for any omitted component.  The iterator is
evaluated as if an index were defined.

\subsubsection{Splitting a Tuple into Multiple Formal Arguments}
\label{Formal_Argument_Declarations_in_a_Tuple}
\index{tuples!formal arguments grouped as}

When multiple formal arguments in a function declaration are grouped
using tuple notation, the actual expression is split into its
components during a function call.  The number of grouped formal
arguments must be equal to the size of the actual tuple expression.
The actual arguments are passed in component order to the formal
arguments.

The syntax of grouped formal arguments is defined
in~\rsec{Function_Definitions}.

\begin{example}
\begin{chapelpre}
\end{chapelpre}
The function
\begin{chapel}
def f(x: int, (y, z): (int, int)) {
  // body
}
\end{chapel}
is defined to take an integer value and a 2-tuple of integer values.
The 2-tuple is split when the function is called into two formals.  A
call may look like the following:
\begin{chapel}
f(1, (2, 3));
\end{chapel}
\begin{chapelpost}
\end{chapelpost}
\begin{chapeloutput}
\end{chapeloutput}
\end{example}

An implicit where clause is created when arguments are grouped using a
tuple notation to ensure that the function is called with an actual
tuple of the correct size.  Grouping arguments in tuples may be
arbitrarily nested.  Functions with tuple-grouped arguments may not be
called using named-argument passing on the tuple-grouped arguments.
In addition, tuple-grouped arguments may not be specified individually
with types or default values (only in aggregate).  They may not be
specified with any qualifier appearing before the group of arguments
(or individual arguments) such as \chpl{inout} or \chpl{type}.  They
may not be followed by \chpl{...} to indicate that there are a
variable number of them.

\begin{example}
\begin{chapelpre}
\end{chapelpre}
The function \chpl{f} defined as
\begin{chapel}
def f((x, (y, z))) {
  writeln((x, y, z));
}
\end{chapel}
is equivalent to the function \chpl{g} defined as
\begin{chapel}
def g(t) where isTuple(t) && t.size == 2 && isTuple(t(2)) && t(2).size == 2 {
  writeln((t(1), t(2)(1), t(2)(2)));
}
\end{chapel}
except without the definition of the argument name \chpl{t}.
\begin{chapelpost}
f((1, (2, 3)));
g((1, (2, 3)));
\end{chapelpost}
\begin{chapeloutput}
(1, 2, 3)
(1, 2, 3)
\end{chapeloutput}
\end{example}

Grouping formal arguments using tuple notation allows a 1-tuple to be
destructured by enclosing a single formal argument in parentheses.
\begin{example}
\begin{chapelpre}
\end{chapelpre}
The empty function
\begin{chapel}
def f((x)) { }
\end{chapel}
accepts a 1-tuple actual with any component type.
\begin{chapelpost}
f(tuple(1));
var y: 1*real;
f(y);
\end{chapelpost}
\begin{chapeloutput}
\end{chapeloutput}
\end{example}

\index{tuples!eliding components}
When splitting a tuple into multiple formal arguments, the arguments
that are grouped using the tuple notation may be omitted.  In this
case, a new argument is not defined for any omitted component.  The
call is evaluated as if an argument were defined.

\subsubsection{Splitting a Tuple via Tuple Expansion}
\label{Tuple_Expansion}
\index{... tuple expansion@\chpl{...} tuple expansion}
\index{tuples!expanding in place}

Tuples can be expanded in place using the following syntax:
\begin{syntax}
tuple-expand-expression:
  ( ... expression )
\end{syntax}
In this expression, the tuple defined by \sntx{expression} is expanded
in place to represent its components.  This can only be used in a
context where a comma-separated list of components is valid.

\begin{example}
\begin{chapelpre}
\end{chapelpre}
Given two 2-tuples
\begin{chapel}
var x1 = (1, 2.0), x2 = ("three", "four");
\end{chapel}
the following statement
\begin{chapel}
var x3 = ((...x1), (...x2));
\end{chapel}
creates the 4-tuple \chpl{x3} with the value \chpl{(1, 2.0, "three",
"four")}.
\begin{chapelpost}
writeln(x3);
\end{chapelpost}
\begin{chapeloutput}
(1, 2.0, three, four)
\end{chapeloutput}
\end{example}

\begin{example}
\begin{chapelpre}
\end{chapelpre}
The following code defines two functions, a function \chpl{first} that
returns the first component of a tuple and a function \chpl{rest} that
returns a tuple containing all of the components of a tuple except for
the first:
\begin{chapel}
def first(t) where isTuple(t) {
  return t(1);
}
def rest(t) where isTuple(t) {
  def helper(first, rest...)
    return rest;
  return helper((...t));
}
\end{chapel}
\begin{chapelpost}
writeln(first((1, 2, 3)));
writeln(rest((1, 2, 3)));
\end{chapelpost}
\begin{chapeloutput}
1
(2, 3)
\end{chapeloutput}
\end{example}

\subsection{Tuple Indexing}
\label{Tuple_Indexing}
\index{tuples!indexing}

A tuple may be accessed by an integral parameter (compile-time
constant) as if it were an array.  Tuples are \emph{1-based} so the
first component in the tuple is accessed by the value \chpl{1}, and so
forth.

\begin{example}
\begin{chapelpre}
\end{chapelpre}
The loop
\begin{chapel}
var tuple = (1, 2.0, "three");
for param i in 1..3 do
  writeln(tuple(i));
\end{chapel}
uses a param loop to output the components of a tuple.
\begin{chapelpost}
\end{chapelpost}
\begin{chapeloutput}
1
2.0
three
\end{chapeloutput}
\end{example}

Homogeneous tuples may be accessed by integral values that are not
necessarily compile-time constants.

\begin{example}
\begin{chapelpre}
\end{chapelpre}
The loop
\begin{chapel}
var tuple = (1, 2, 3);
for i in 1..3 do
  writeln(tuple(i));
\end{chapel}
uses a serial loop to output the components of a homogeneous tuple.
Since the index is not a compile-time constant, this would result in
an error were tuple not homogeneous.
\begin{chapelpost}
\end{chapelpost}
\begin{chapeloutput}
1
2
3
\end{chapeloutput}
\end{example}

\begin{rationale}
Non-homogeneous tuples can only be accessed by compile-time constants
so that the type of the expression is statically known.
\end{rationale}

\subsection{Tuple Operators}
\label{Tuple_Operators}
\index{tuples!operators}

\subsubsection{Unary Operators}
\label{Tuple_Unary_Operators}
%\index{\+ (unary)@\chpl{\+} (unary)}
%\index{\- (unary)@\chpl{\-} (unary)}
%\index{\~ (unary)@\chpl{\~} (unary)}
%\index{\! (unary)@\chpl{\!} (unary)}

The unary operators \chpl{\+}, \chpl{\-}, \chpl{\~}, and \chpl{\!} are
overloaded on tuples by applying the operator to each argument component
and returning the results as a new tuple.

The size of the result tuple is the same as the size of the
argument tuple. The type of each result component is the result
type of the operator when applied to the corresponding argument component.

\subsubsection{Binary Operators}
\label{Tuple_Binary_Operators}
%\index{\+@\chpl{\+}}
%\index{\-@\chpl{\-}}
%\index{\*@\chpl{\*}}
%\index{\/@\chpl{\/}}
%\index{\%@\chpl{\%}}
%\index{\*\*@\chpl{\*\*}}
%\index{\&@\chpl{\&}}
%\index{\|@\chpl{\|}}
%\index{\^@\chpl{\^}}
%\index{\<\<@\chpl{\<\<}}
%\index{\>\>@\chpl{\>\>}}

The binary operators \chpl{\+}, \chpl{\-}, \chpl{\*}, \chpl{\/}, \chpl{\%},
\chpl{\*\*}, \chpl{\&}, \chpl{\|}, \chpl{\^}, \chpl{\<\<}, and \chpl{\>\>}
are overloaded on tuples by applying them to pairs of the respective
argument components and returning the results as a new tuple.
The sizes of the two argument tuples must be the same.

The size of the result tuple is the same as the argument tuples.  The
type of each result component is the result type of the operator when
applied to the corresponding pair of the argument components.

\begin{example}
\begin{chapelpre}
\end{chapelpre}
The code
\begin{chapel}
var x = (1, 1, 1) + (2, 2.0, "2");
\end{chapel}
creates a 3-tuple of an int, a real and a string with the value \chpl{(3, 3.0, "12")}.
\begin{chapelpost}
writeln(x);
\end{chapelpost}
\begin{chapeloutput}
(3, 3.0, 12)
\end{chapeloutput}
\end{example}


\subsubsection{Relational Operators}
\label{Tuple_Relational_Operators}
%\index{\>@\chpl{\>}}
%\index{\>\=@\chpl{\>\=}}
%\index{\<@\chpl{\<}}
%\index{\<\=@\chpl{\<\=}}
%\index{\=\=@\chpl{\=\=}}
%\index{\!\=@\chpl{\!\=}}

% (\rsec{Relational_Operators})

The relational operators \chpl{\>}, \chpl{\>\=}, \chpl{\<}, \chpl{\<\=},
\chpl{\=\=}, and \chpl{\!\=} are defined over tuples of matching size.
They return a single boolean value indicating whether the two
arguments satisfy the corresponding relation.

The operators \chpl{\>}, \chpl{\>\=}, \chpl{\<}, and \chpl{\<\=}
check the corresponding lexicographical order
based on pair-wise comparisons between the argument tuples' components.
%based on comparisons between pairs of the respective
%components of the arguments.
The operators \chpl{\=\=} and \chpl{\!\=} check whether
the two arguments are pair-wise equal or not.
The relational operators on tuples may be short-circuiting, i.e.
they may execute only the pair-wise comparisons that are necessary
to determine the result.

\begin{example}
\begin{chapelpre}
\end{chapelpre}
The code
\begin{chapel}
var x = (1, 1, 0) > (1, 0, 1);
\end{chapel}
creates a variable initialized to \chpl{true}.  After comparing the
first components and determining they are equal, the second components
are compared to determine that the first tuple is greater than the
second tuple.
\begin{chapelpost}
writeln(x);
\end{chapelpost}
\begin{chapeloutput}
true
\end{chapeloutput}
\end{example}

\subsection{Predefined Functions and Methods on Tuples}
\index{tuples!predefined functions}

\begin{protohead}
def $Tuple$.size param
\end{protohead}
\begin{protobody}
Returns the size of the tuple.
\end{protobody}

\begin{protohead}
def isHomogeneousTuple(t: $Tuple$) param
\end{protohead}
\begin{protobody}
Returns true if \chpl{t} is a homogeneous tuple; otherwise false.
\end{protobody}

\begin{protohead}
def isTuple(t: $Tuple$) param
\end{protohead}
\begin{protobody}
Returns true if \chpl{t} is a tuple; otherwise false.
\end{protobody}

\begin{protohead}
def isTupleType(type t) param
\end{protohead}
\begin{protobody}
Returns true if \chpl{t} is a tuple of types; otherwise false.
\end{protobody}

\begin{protohead}
def max(type t) where isTupleType(t)
\end{protohead}
\begin{protobody}
Returns a tuple of type \chpl{t} with each component set to the maximum
value that can be stored in its position.
\end{protobody}

\begin{protohead}
def min(type t) where isTupleType(t)
\end{protohead}
\begin{protobody}
Returns a tuple of type \chpl{t} with each component set to the minimum
value that can be stored in its position.
\end{protobody}

